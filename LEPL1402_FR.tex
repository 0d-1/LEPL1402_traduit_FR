\UseRawInputEncoding
\documentclass[11pt,a4paper]{book}

\usepackage[utf8]{inputenc}
\usepackage[T1]{fontenc}
\usepackage[french]{babel}
\usepackage{amsmath,amssymb}
\usepackage{graphicx}
\usepackage{hyperref}
\usepackage{url}
\usepackage{listings}
\usepackage{xcolor}
\usepackage{geometry}

\geometry{margin=2.5cm}

\hypersetup{
  pdftitle={Concepts de Programmation avec Java},
  pdfauthor={Sébastien Jodogne, Ramin Sadre, Pierre Schaus},
  colorlinks=true,
  linkcolor=blue,
  urlcolor=blue,
  citecolor=blue
}

\definecolor{codebg}{RGB}{248,248,255}
\definecolor{codeframe}{RGB}{200,200,200}
\definecolor{codekeyword}{RGB}{0,102,204}
\definecolor{codecomment}{RGB}{0,128,0}
\definecolor{codestring}{RGB}{163,21,21}

\lstset{
  basicstyle=\ttfamily\small,
  keywordstyle=\color{codekeyword}\bfseries,
  commentstyle=\color{codecomment}\itshape,
  stringstyle=\color{codestring},
  backgroundcolor=\color{codebg},
  showstringspaces=false,
  tabsize=2,
  breaklines=true,
  frame=single,
  rulecolor=\color{codeframe}
}

\title{Concepts de Programmation avec Java}
\author{Sébastien Jodogne \and Ramin Sadre \and Pierre Schaus}
\date{10 janvier 2025}

\begin{document}

\maketitle

\tableofcontents

\chapter{De Python à Java}

Le chapitre~1 de ce livre s'adresse aux étudiants et passionnés qui sont déjà
familiers avec les bases de la programmation en Python, c'est-à-dire qui savent
utiliser des variables, des listes, des fonctions et de simples objets de
données. Une connaissance approfondie de la programmation orientée objet n'est
pas requise.

L'objectif des sections suivantes est de vous rendre rapidement familier avec
les différences importantes entre Python et Java, ainsi qu'avec les mécanismes
orientés objet de base de Java. Les sujets plus avancés, tels que les
interfaces, les classes abstraites ou les fonctions lambda, seront abordés dans
les parties suivantes du livre.

\section{Votre premier programme Java avec IntelliJ}

\subsection{Installation d'IntelliJ}

Vous avez peut-être déjà utilisé un environnement de développement intégré
(IDE) pour écrire des programmes en Python. Dans ce cours, nous ferons la même
chose pour programmer en Java : nous utiliserons la « Community Edition »
gratuite de IntelliJ IDEA (simplement appelée « IntelliJ » par la suite). Vous
pouvez télécharger l'installateur depuis
\url{https://www.jetbrains.com/idea/download/}
(descendez dans la page pour trouver la Community Edition gratuite, vous n'avez
pas besoin de l'édition commerciale Ultimate). Lancez l'installateur et
suivez les instructions.

La deuxième chose dont vous aurez besoin pour programmer en Java est un
\emph{Java Development Kit} (JDK). Un JDK est un ensemble logiciel qui contient
les outils nécessaires pour construire et exécuter des programmes Java. Le JDK
comprend également une bibliothèque très, très vaste de classes utiles pour
toutes sortes de tâches de programmation. Vous pouvez consulter le contenu de
cette bibliothèque ici :
\url{https://docs.oracle.com/en/java/javase/21/docs/api/index.html}.

Heureusement, IntelliJ peut automatiquement télécharger le JDK pour vous lorsque
vous créez un nouveau projet, donc vous n'avez pas à vous soucier du JDK pour
l'instant. Mais si plus tard vous souhaitez écrire une application Java sur un
ordinateur sans IntelliJ, vous devrez télécharger manuellement le JDK depuis
\url{https://openjdk.org/} et l'installer.

\subsection{Création d'un nouveau projet}

Démarrez IntelliJ. Une fenêtre s'ouvre et vous permet de créer un nouveau
projet. Cliquez sur le bouton correspondant et vous devriez voir une fenêtre
semblable à celle-ci.

Pour créer un nouveau projet, vous devez entrer un nom de projet (dans le champ
\textit{Name}) et un emplacement sur votre disque où vous souhaitez stocker le
projet (dans le champ \textit{Location}). Laissez les autres champs
\textit{Language}, \textit{Build system} et \textit{Add sample code}
tels qu'affichés sur l'image. Il reste cependant quelque chose à faire pour le
champ \textit{JDK} : comme vous pouvez le voir sur l'image, la version 21 du
JDK (ainsi que d'autres versions) était déjà installée sur mon ordinateur. Si
vous n'avez pas encore installé de JDK sur votre machine, ouvrez la liste
déroulante et choisissez \textit{Download JDK...} comme sur l'image ci-dessous.

Une petite fenêtre apparaît alors et vous permet de sélectionner la version du
JDK à télécharger et installer.

Sélectionnez la version 21 fournie par le fournisseur \emph{Oracle OpenJDK}
(en réalité, toute version plus récente que 17 convient pour ce livre). Vous
pouvez conserver l'emplacement proposé par IntelliJ. Cliquez sur le bouton
\textit{Download} et terminez l'installation du JDK. Une fois tout prêt, vous
pouvez finalement créer votre premier projet Java. IntelliJ ouvrira
normalement automatiquement le nouveau projet et affichera la fenêtre
principale.

Dans la partie gauche de la fenêtre, vous voyez la structure du projet. Comme
nous avons coché \textit{Add sample code} lors de la création du projet,
IntelliJ a déjà créé un répertoire \texttt{src} contenant un fichier :
\texttt{Main.java} (l'extension \texttt{.java} peut ne pas être affichée).
Lorsque vous double-cliquez sur ce fichier, son contenu apparaît dans l'éditeur
à droite.

Cliquez sur le triangle orienté vers la droite dans le coin supérieur droit
pour lancer le programme. Une nouvelle vue apparaît en bas de la fenêtre avec
la sortie du programme.

\subsection{À quoi ressemblent les programmes Java ?}

Voici le code source du programme d'exemple automatiquement créé par IntelliJ
dans votre projet :

\begin{lstlisting}[language=Java]
public class Main {
    public static void main(String[] args) {
        System.out.println("Hello world!");
    }
}
\end{lstlisting}

Et voici à quoi ressemblerait un programme Python équivalent :

\begin{lstlisting}[language=Python]
print('Hello world!')
\end{lstlisting}

Pourquoi le code Java semble-t-il plus compliqué que le code Python ? Tout
d'abord, contrairement à Python, Java ne permet pas d'écrire une instruction
comme \texttt{print('Hello world!')} directement dans un fichier source. En
Java, toutes les instructions DOIVENT se trouver à l'intérieur d'une méthode et
toutes les méthodes DOIVENT se trouver à l'intérieur d'une classe. Dans notre
exemple, l'instruction \texttt{System.out.println("Hello world!")} se trouve
dans la méthode \texttt{main()} et cette méthode se trouve dans la classe
\texttt{Main}. Bien sûr, une classe Java peut contenir plusieurs méthodes, et
un programme Java peut contenir plusieurs classes.

Vous avez probablement déjà entendu parler des classes et des méthodes en
Python, et vous vous souvenez peut-être que les classes servent à décrire des
objets et que les méthodes servent à travailler avec ces objets. Dans notre
simple exemple Java, nous n'avons pas besoin d'objets ni de toutes les choses
compliquées qui vont avec (constructeurs, héritage, etc.). Le mot-clé
\texttt{static} dans la ligne
\texttt{public static void main(String[] args)} indique que la méthode
\texttt{main()} se comporte davantage comme une fonction « traditionnelle » en
Python et non comme une méthode d'instance. En effet, aucun objet n'est
nécessaire pour exécuter une méthode statique comme \texttt{main()}. Nous en
apprendrons plus à ce sujet plus tard.

La deuxième chose que vous avez peut-être remarquée est le mot \texttt{public},
présent deux fois dans les deux premières lignes du code :

\begin{lstlisting}[language=Java]
public class Main {
    public static void main(String[] args) {
\end{lstlisting}

Le mot \texttt{public} dans la première ligne indique que la classe
\texttt{Main} peut être utilisée par d'autres. Ce n'est pas strictement
nécessaire pour ce programme simple et, en fait, notre programme fonctionnera
toujours si vous le supprimez (essayez !). Cependant, il y a un point important
à connaître sur les classes \texttt{public} : si une classe est marquée
\texttt{public}, le fichier source qui la contient doit avoir le même nom que
la classe. C'est pour cela que le fichier s'appelle \texttt{Main.java} et que
la classe \texttt{public} dans ce fichier s'appelle \texttt{Main} (essayez de
changer le nom de la classe et voyez ce qui se passe !). À part cela, le nom
\texttt{Main} n'a aucune signification particulière en Java. Notre programme
fonctionnerait encore si nous renommions la classe en \texttt{Catweazle} ou
\texttt{Cinderella}, à condition de renommer aussi le fichier. Notez toutefois
que tous les noms de classes en Java (publiques ou non) commencent par une
majuscule.

Le \texttt{public} de la deuxième ligne est beaucoup plus important dans notre
exemple. Un programme Java ne peut être exécuté que s'il contient une méthode
\texttt{main()} qui est \texttt{public} et \texttt{static}. Supprimez
\texttt{public} ou \texttt{static} dans la deuxième ligne et observez ce qui se
passe lorsque vous essayez d'exécuter le programme. De manière générale, un
programme Java commence toujours dans une méthode
\texttt{public static void main(String[] args)}. Si votre programme contient
plusieurs classes avec une méthode \texttt{main}, vous devez indiquer à
IntelliJ laquelle lancer.

Avec ces connaissances, pouvez-vous deviner ce qu'affiche le programme
suivant ?

\begin{lstlisting}[language=Java]
public class Main {
    static void printHello() {
        System.out.print("How do ");
        System.out.println("you do, ");
    }

    public static void main(String[] args) {
        printHello();
        System.out.println("fellow kids?");
    }
}
\end{lstlisting}

(Au fait, avez-vous remarqué la différence entre
\texttt{System.out.print} et \texttt{System.out.println} ?)

Un fichier \texttt{.java} peut contenir plus d'une classe ; toutefois, une
seule de ces classes peut être \texttt{public}. Voici l'exemple précédent avec
deux classes :

\begin{lstlisting}[language=Java]
class MyOtherClass {
    static void printHello() {
        System.out.print("How do ");
        System.out.println("you do, ");
    }
}

public class Main {
    public static void main(String[] args) {
        MyOtherClass.printHello();
        System.out.println("fellow kids?");
    }
}
\end{lstlisting}

Vous pouvez accéder au contenu statique d'une classe depuis une autre classe en
utilisant le nom de la classe, comme démontré avec
\texttt{MyOtherClass.printHello()} dans l'exemple.

\subsection{Types}

Vous savez peut-être déjà que Python est un langage fortement typé. Cela
signifie que toutes les « choses » en Python ont un type spécifique. Vous
pouvez le constater en entrant les instructions suivantes dans l'interpréteur
Python :

\begin{lstlisting}[language=Python]
>>> type("hello")
<class 'str'>
>>> type(1234)
<class 'int'>
>>> type(1234.5)
<class 'float'>
>>> type(True)
<class 'bool'>
\end{lstlisting}

Java est lui aussi un langage fortement typé. Cependant, il y a une grande
différence avec Python : Java est également un langage à typage statique. Sans
entrer dans tous les détails, cela signifie qu'en Java, vous devez, la plupart
du temps, indiquer pour chaque variable de votre programme quel type de choses
elle peut contenir.

Voici un simple programme Python qui calcule et affiche l'aire d'un carré :

\begin{lstlisting}[language=Python]
def calculateArea(side):
    return side * side

def printArea(message, side):
    area = calculateArea(side)
    print(message)
    print(area)

t = 3 + 4
printArea("Area of square", t)
\end{lstlisting}

Et voici le programme Java équivalent :

\begin{lstlisting}[language=Java]
public class Main {
    static int calculateArea(int side) {
        return side * side;
    }

    static void printArea(String message, int side) {
        int area = calculateArea(side);
        System.out.println(message);
        System.out.println(area);
    }

    public static void main(String[] args) {
        int t = 3 + 4;
        printArea("Area of square", t);
    }
}
\end{lstlisting}

Voyons ce qui se passe avec les types dans ce code Java :

\begin{itemize}
  \item La ligne \lstinline!int calculateArea(int side)! indique que la
        méthode \lstinline!calculateArea()! a un paramètre \texttt{side} de
        type \texttt{int}. De plus, le \texttt{int} au début indique que cette
        méthode ne peut retourner qu'une valeur de type \texttt{int}. C'est le
        \emph{type de retour} de la méthode.
  \item La ligne \lstinline!void printArea(String message, int side)!
        définit que la méthode \lstinline!printArea()! a un paramètre
        \texttt{message} de type \texttt{String} et un paramètre
        \texttt{side} de type \texttt{int}. La méthode ne retourne rien ;
        elle a donc le type de retour spécial \texttt{void}.
  \item À l'intérieur de \lstinline!printArea()!, la ligne
        \lstinline!int area = calculateArea(side)! indique que la variable
        \texttt{area} est de type \texttt{int}.
  \item (Exercice : examinez les types visibles dans la méthode
        \lstinline!main()!. Nous expliquerons plus tard pourquoi cette méthode
        a toujours un paramètre \texttt{args}.)
\end{itemize}

IntelliJ utilise un outil particulier appelé \emph{compilateur Java}, qui
vérifie soigneusement l'absence d'erreurs de type dans votre programme, c'est-à-dire
que vous n'avez pas commis d'erreurs dans les types des variables, des
paramètres de méthodes et des types de retour. Contrairement à Python, cette
vérification de types est faite avant l'exécution du programme. Vous ne pouvez
même pas démarrer un programme Java qui contient des erreurs de type !

Voici quelques exemples contenant des erreurs de type. Pouvez-vous trouver les
erreurs ?

\begin{itemize}
  \item \lstinline!int t = "Hello";!
  \item \lstinline!boolean t = calculateArea(3);!
  \item \lstinline!printArea(5, "Size of square");!
        (cet exemple illustre pourquoi il est plus facile de trouver des bugs
        en Java qu'en Python)
\end{itemize}

\section{Le compilateur Java et les fichiers \texttt{class}}

Dans la section précédente, nous avons mentionné qu'un outil particulier, le
compilateur Java, vérifie votre programme à la recherche d'erreurs de type.
Cette vérification fait partie d'une autre différence fondamentale entre Python
et Java. Python est un langage interprété. Cela signifie que lorsque vous
lancez un programme écrit en Python dans un IDE ou en ligne de commande avec

\begin{lstlisting}
> python myprogram.py
\end{lstlisting}

l'interpréteur Python effectue les opérations suivantes :

\begin{enumerate}
  \item Charger le fichier \texttt{myprogram.py},
  \item Effectuer quelques vérifications pour s'assurer que votre programme ne
        contient pas d'erreurs de syntaxe, par exemple
        \texttt{print('Hello')))))},
  \item Exécuter votre programme.
\end{enumerate}

Java, en tant que langage compilé, fonctionne différemment. Pour exécuter un
programme Java, une étape supplémentaire est effectuée avant que votre
programme puisse être exécuté :

\begin{enumerate}
  \item Tout d'abord, le code Java doit être compilé. C'est le travail du
        compilateur Java, un outil fourni avec le JDK. Le compilateur :
        \begin{itemize}
          \item vérifie que votre code source est un programme Java
                bien formé. Ce processus inclut la vérification de types
                décrite précédemment ;
          \item traduit votre code source Java en une représentation plus
                compacte, plus facile à traiter par l'ordinateur. Cette
                représentation compacte est appelée \emph{fichier de classe}
                (\texttt{.class}). Un fichier de ce type est créé par classe
                de votre programme. Dans IntelliJ, vous pouvez trouver les
                fichiers \texttt{.class} générés dans le répertoire
                \texttt{out} de votre projet.
        \end{itemize}
  \item Si la compilation de votre code a réussi, la \emph{Java Virtual
        Machine} (JVM) est démarrée. La JVM est un programme spécial qui peut
        charger et exécuter des fichiers de classe. La JVM n'a pas besoin du
        code source (\texttt{.java}) pour exécuter votre programme, puisque
        les fichiers \texttt{.class} contiennent toutes les informations
        nécessaires. Quand vous développez un logiciel pour d'autres, ce sont
        généralement les fichiers \texttt{.class} que vous leur fournissez,
        pas le code source.
\end{enumerate}

IntelliJ lance le compilateur Java et démarre la JVM pour vous lorsque vous
appuyez sur le bouton de démarrage (vert), mais il est parfaitement possible
de faire cela à la main en ligne de commande, sans IDE :

\begin{lstlisting}
> javac Main.java   # javac est le compilateur, fourni avec le JDK.
                    # Il génère le fichier Main.class

> java Main         # cette commande démarre la JVM avec votre classe Main
\end{lstlisting}

\section{Bases}

\subsection{Types primitifs}

Comme expliqué, Java exige que vous précisiez le type de toutes les variables
(y compris les paramètres de méthodes) et les types de retour de toutes les
méthodes. Java distingue les \emph{types primitifs} et les types complexes,
comme les tableaux et les objets. Les types primitifs sont utilisés pour les
nombres (entiers et réels), les valeurs booléennes (\texttt{true} et
\texttt{false}) et les caractères individuels (a, b, etc.). Il existe
cependant plusieurs types de nombres. Le tableau ci-dessous présente tous les
types primitifs :

\begin{center}
\begin{tabular}{lll}
\textbf{Type} & \textbf{Valeurs possibles} & \textbf{Exemple} \\
\hline
\texttt{int} & $-2^{31}..2^{31}-1$ & \lstinline!int a = 3;! \\
\texttt{long} & $-2^{63}..2^{63}-1$ & \lstinline!long a = 3;! \\
\texttt{short} & $-2^{15}..2^{15}-1$ & \lstinline!short a = 3;! \\
\texttt{byte} & $-2^{7}..2^{7}-1$ & \lstinline!byte a = 3;! \\
\texttt{float} & $\approx 1.4 \cdot 10^{-45}..3.4 \cdot 10^{38}$ &
  \lstinline!float a = 3.45f;! \\
\texttt{double} & $\approx 4.9 \cdot 10^{-324}..1.7 \cdot 10^{308}$ &
  \lstinline!double a = 3.45;! \\
\texttt{char} & $0..2^{16}-1$ & \lstinline!char a = 'X';! \\
\texttt{boolean} & \texttt{true}, \texttt{false} &
  \lstinline!boolean a = true;! \\
\end{tabular}
\end{center}

Comme vous pouvez le voir, chaque type primitif a un domaine de valeurs limité.
Par exemple, une variable de type \texttt{int} ne peut être utilisée que pour
des entiers compris entre $-2^{31}$ et $2^{31}-1$. Si vous ne respectez pas le
domaine d'un type, des choses très étranges se produiront dans votre programme
! Essayez ce code dans IntelliJ (copiez-le dans la méthode
\texttt{main()}) :

\begin{lstlisting}[language=Java]
int a = 123456789;
int b = a * 100000; // Ce résultat est trop grand pour le type int !
System.out.println(b); // Qu'obtenez-vous ici ?
\end{lstlisting}

Pour la plupart des exemples de ce livre, il suffira d'utiliser \texttt{int}
(pour les entiers) et \texttt{float} (pour les réels). Les types
\texttt{long} et \texttt{double} offrent un domaine plus large et davantage de
précision, mais ils sont plus lents et votre programme consommera plus de
mémoire.

Java prend en charge les opérations arithmétiques usuelles sur les types
numériques : \texttt{+} (addition), \texttt{-} (soustraction),
\texttt{*} (multiplication), \texttt{/} (division) et \texttt{\%}
(modulo). Il existe aussi des opérateurs pour manipuler les valeurs entières au
niveau des bits (par exemple, le décalage à gauche \texttt{<<} et le « et »
binaire \texttt{\&}), mais nous ne les détaillerons pas ici.

Le type \texttt{char} est utilisé pour manipuler des caractères individuels
(lettres, chiffres, etc.) :

\begin{lstlisting}[language=Java]
char c = 'a';
\end{lstlisting}

Vous pouvez vous demander pourquoi ce type apparaît dans le tableau comme ayant
des valeurs entre 0 et 65535. C'est parce que Java représente les caractères
par des nombres, selon le standard Unicode. Par conséquent, vous pouvez réaliser
certaines opérations arithmétiques simples sur les caractères :

\begin{lstlisting}[language=Java]
char c = 'a';
c++;
System.out.println(c); // affiche 'b'
\end{lstlisting}

\subsection{Conversions de type (cast)}

Java effectue des conversions automatiques entre valeurs de différents types si
le type de destination est « assez grand » pour contenir le résultat. On parle
de \emph{cast automatique}. C'est pourquoi les deux instructions suivantes sont
autorisées :

\begin{lstlisting}[language=Java]
float a = 34;        // valeur int 34 convertie en float 34.0f
float b = 6 * 4.5f;  // int * float donne un float
\end{lstlisting}

Mais ceci n'est pas autorisé :

\begin{lstlisting}[language=Java]
int a = 4.5f;          // Erreur ! pas de cast automatique float -> int
float b = 4.5f * 6.7;  // Erreur ! float * double donne un double
\end{lstlisting}

Vous pouvez forcer la conversion en effectuant un cast explicite, mais le
résultat sera moins précis, voire erroné dans certaines situations :

\begin{lstlisting}[language=Java]
int a = (int) 4.5f;           // donne 4
float b = (float) (4.5f * 6.7);
\end{lstlisting}

La classe Java \texttt{Math} fournit un grand nombre de méthodes pour travailler
avec des nombres de différents types. Elle définit aussi des constantes utiles
comme \texttt{Math.PI}. Exemple :

\begin{lstlisting}[language=Java]
double area = 123.4;
double radius = Math.sqrt(area / Math.PI);

System.out.println("Area of disk: " + area);
System.out.println("Radius of disk: " + radius);
\end{lstlisting}

\subsection{Qu'est-ce qu'une variable ? Modèle mental}

Pour les variables de type primitif, vous pouvez imaginer que chaque fois que
votre programme atteint une ligne où une variable est déclarée, la JVM réserve
une petite partie de la mémoire principale (RAM) de votre ordinateur pour
stocker la valeur de cette variable.

Lorsque vous affectez le contenu d'une variable à une autre, la valeur est
copiée. Il en va de même pour les paramètres de méthodes : lorsqu'une méthode
est appelée avec des arguments, leurs valeurs sont copiées dans les variables
paramètres de la méthode appelée.

Considérez le programme suivant :

\begin{lstlisting}[language=Java]
public class Main {
    static void f(int x) {
        x = x + 1;
    }

    public static void main(String[] args) {
        int i = 3;
        f(i);
        System.out.println(i);
    }
}
\end{lstlisting}

Ce programme affiche \texttt{3}, car lors de l'appel à \texttt{f(i)}, le
contenu de \texttt{i} est copié dans le paramètre \texttt{x}. Même si la
méthode modifie \texttt{x}, la variable \texttt{i} conserve la valeur 3.

Il est illégal d'utiliser une variable locale (déclarée dans une méthode) avant
de lui avoir affecté une valeur :

\begin{lstlisting}[language=Java]
public static void main(String[] args) {
    int a = 2;
    int b;
    int c;

    int d = a * 3;   // OK

    b = 3;
    int e = b * 3;   // OK

    int f = c * 3;   // Erreur ! "c" n'a pas été initialisée.
}
\end{lstlisting}

\subsection{Variables de classe}

Dans nos exemples, toutes les variables étaient des paramètres ou des variables
locales. Ces variables ne « vivent » que pendant l'exécution de la méthode où
elles sont définies. Il est également possible d'avoir des variables qui
« vivent » en dehors de toute méthode : ce sont les \emph{variables de classe}
(ou variables statiques), car elles appartiennent à la classe et non à une
méthode. Comme pour les méthodes statiques, on les marque avec le mot-clé
\texttt{static} :

\begin{lstlisting}[language=Java]
public class Main {

    static int a = 3; // variable de classe

    static void increment() {
        a += 5; // équivaut à a = a + 5
    }

    public static void main(String[] args) {
        increment();
        System.out.println(a);
    }
}
\end{lstlisting}

Contrairement aux variables locales, les variables de classe sont
automatiquement initialisées : à 0 pour les nombres, \texttt{false} pour les
booléens. Ce code est donc accepté :

\begin{lstlisting}[language=Java]
public class Main {

    static int a; // équivaut à a = 0

    static void increment() {
        a += 5;
    }

    public static void main(String[] args) {
        increment();
        System.out.println(a);
    }
}
\end{lstlisting}

Attention lorsque vous avez des variables de classe et des variables locales ou
paramètres portant le même nom :

\begin{lstlisting}[language=Java]
public class Main {

    static int a = 3;

    static void increment(int a) {
        a += 5; // ici, on modifie le paramètre, pas la variable de classe
    }

    public static void main(String[] args) {
        increment(10);
        System.out.println(a);
    }
}
\end{lstlisting}

Le paramètre \texttt{a} masque (\emph{shadow}) la variable de classe
\texttt{a} dans le corps de la méthode \texttt{increment}. Si vous devez malgré
tout accéder à la variable de classe, vous pouvez écrire :

\begin{lstlisting}[language=Java]
public class Main {

    static int a = 3;

    static void increment(int a) {
        Main.a += 5; // on vise explicitement la variable de classe
    }

    public static void main(String[] args) {
        increment(10);
        System.out.println(a);
    }
}
\end{lstlisting}

\subsection{Tableaux}

Si vous avez besoin d'un certain nombre de variables de même type primitif, il
peut être utile d'utiliser un \emph{tableau} (\texttt{array}) plutôt que des
variables séparées. Les tableaux sont similaires aux listes en Python, avec
d'importantes différences :

\begin{itemize}
  \item Vous devez spécifier la taille lors de la création.
  \item Tous les éléments ont le même type.
\end{itemize}

\begin{lstlisting}[language=Java]
int[] a = new int[4]; // tableau de 4 entiers
\end{lstlisting}

Les éléments d'un tableau sont automatiquement initialisés (à 0 pour
\texttt{int}, etc.) :

\begin{lstlisting}[language=Java]
int[] a = new int[4]; // tous les éléments valent 0
a[2] = 5;
int b = a[1] + a[2];
System.out.println(b); // affiche 5
\end{lstlisting}

La taille d'un tableau est fixe : on ne peut pas ajouter ou supprimer des
éléments (contrairement aux listes Python). Plus tard, nous verrons la classe
\texttt{ArrayList}, plus flexible.

\subsection{Modèle mental pour les tableaux}

Une variable de type tableau ne contient pas directement les éléments ; elle
contient une \emph{référence} vers les données du tableau. Lorsqu'on écrit :

\begin{lstlisting}[language=Java]
int[] a = new int[4];
int[] b = a;
\end{lstlisting}

\texttt{a} et \texttt{b} référencent le même tableau. Modifier via
\texttt{b} affecte aussi ce que l'on voit via \texttt{a} :

\begin{lstlisting}[language=Java]
int[] a = new int[4];
int[] b = a;
b[2] = 5;
System.out.println(a[2]); // affiche 5
\end{lstlisting}

Même chose lorsqu'un tableau est passé en paramètre :

\begin{lstlisting}[language=Java]
public class Main {

    static void five(int[] x) {
        x[2] = 5;
    }

    public static void main(String[] args) {
        int[] a = new int[4];
        five(a);
        System.out.println(a[2]); // affiche 5
    }
}
\end{lstlisting}

\subsection{Initialisation d'un tableau}

On peut créer et initialiser un tableau en une seule étape :

\begin{lstlisting}[language=Java]
int[] a = { 2, 5, 6, -3 };
\end{lstlisting}

Ceci équivaut à créer le tableau puis affecter chaque élément un par un.
Attention : cette syntaxe compacte n'est autorisée que lors de la déclaration.
Pour réaffecter un tableau plus tard, on doit écrire :

\begin{lstlisting}[language=Java]
int[] a = { 2, 5, 6, -3 };
a = new int[]{ 1, 9, 3, 4 };
\end{lstlisting}

\subsection{Tableaux multidimensionnels}

Les tableaux peuvent avoir plusieurs dimensions. Par exemple, un tableau
bidimensionnel peut représenter une matrice :

\begin{lstlisting}[language=Java]
int[][] a = new int[3][5]; // matrice 3x5
a[2][4] = 5;
\end{lstlisting}

Un \texttt{int[3][5]} est en fait un tableau de 3 références vers des tableaux
d'entiers de taille 5.

\begin{lstlisting}[language=Java]
int[][] a = new int[3][5];
int[] b = a[0]; // b référence la première ligne (5 éléments)
b[3] = 7;
System.out.println(a[0][3]); // même élément
\end{lstlisting}

On peut aussi initialiser directement :

\begin{lstlisting}[language=Java]
// matrice identité 3x3
int[][] a = {
    { 1, 0, 0 },
    { 0, 1, 0 },
    { 0, 0, 1 }
};
\end{lstlisting}

\subsection{Tableaux partiellement initialisés}

Il est possible de ne spécifier que la première dimension :

\begin{lstlisting}[language=Java]
int[][] a = new int[3][];
\end{lstlisting}

Ici, chaque élément de \texttt{a} est initialisé à \texttt{null}. On les
initialise ensuite :

\begin{lstlisting}[language=Java]
int[][] a = new int[3][];
a[0] = new int[5];
a[1] = new int[5];
a[2] = new int[2]; // autorisé : pas obligé d'avoir la même taille
\end{lstlisting}

\subsection{Tableaux et variables de classe}

Les variables de type tableau peuvent aussi être des variables de classe :

\begin{lstlisting}[language=Java]
public class Main {

    static int[] a; // initialisée automatiquement à null

    public static void main(String[] args) {
        // compile, mais lancera une erreur à l'exécution
        System.out.println(a[2]);
    }
}
\end{lstlisting}

La valeur \texttt{null} représente une référence invalide.

\subsection{Boucles \texttt{while}}

La boucle \texttt{while} en Java est très proche de celle de Python :

\begin{lstlisting}[language=Java]
int sum = 0;
int i = 0;
while (i < 10) {
    sum += i;
    System.out.println("Nearly there");
    i++;
}
System.out.println("The sum is " + sum);
\end{lstlisting}

Attention : si le corps de la boucle contient plus d'une instruction, il doit
être entouré d'accolades \texttt{\{...\}}. Il est fortement recommandé de
toujours utiliser les accolades, même pour une seule instruction.

\subsection{Boucles \texttt{for} simples}

La forme simplifiée de \texttt{for} (\emph{for-each}) est pratique pour
parcourir les éléments d'un tableau :

\begin{lstlisting}[language=Java]
int[] myArray = new int[]{ 2, 5, 6, -3 };
int sum = 0;
for (int elem : myArray) {
    sum += elem;
}
System.out.println("The sum is " + sum);
\end{lstlisting}

\subsection{Boucles \texttt{for} complexes}

La forme complète donne un contrôle total :

\begin{lstlisting}[language=Java]
int sum = 0;
for (int i = 0; i < 10; i++) {
    sum += i;
    System.out.println("Nearly there");
}
System.out.println("The sum is " + sum);
\end{lstlisting}

Exemple : somme d'un élément sur deux dans un tableau :

\begin{lstlisting}[language=Java]
int[] myArray = new int[]{ 2, 5, 6, -3, 4, 1 };
int sum = 0;
for (int i = 0; i < myArray.length; i += 2) {
    sum += myArray[i];
}
System.out.println("The sum is " + sum);
\end{lstlisting}

\subsection{Arrêter une boucle, sauter des itérations}

\texttt{break} permet de sortir de la boucle, \texttt{continue} de passer
directement à l'itération suivante :

\begin{lstlisting}[language=Java]
for (int i = 0; i < 10; i++) {
    if (i == 5) {
        continue;
    }
    // ...
}
\end{lstlisting}

À n'utiliser que s'ils rendent le code plus lisible.

\subsection{Instructions \texttt{if/else}}

L'instruction \texttt{if} fonctionne comme en Python. Exemple :

\begin{lstlisting}[language=Java]
int[] myArray = new int[]{ 2, -5, 6, 0, -4, 1 };
int countNegative = 0;
int countPositive = 0;
for (int elem : myArray) {
    if (elem < 0) {
        countNegative++;
    } else if (elem > 0) {
        countPositive++;
    } else {
        System.out.println("Value zero found");
    }
}
System.out.println("The number of negative values is " + countNegative);
System.out.println("The number of positive values is " + countPositive);
\end{lstlisting}

Toujours penser aux accolades.

\subsection{Opérateurs de comparaison et logiques}

\begin{lstlisting}[language=Java]
boolean b1 = 3 < 4;     // <, >, <=, >=, ==, !=
boolean b2 = !b1;       // négation
boolean b3 = b1 && b2;  // et logique
boolean b4 = b1 || b2;  // ou logique
\end{lstlisting}

\subsection{Instruction \texttt{switch}}

Plutôt qu'une longue suite de \texttt{if/else if} :

\begin{lstlisting}[language=Java]
int x = 0, y = 0;
char[] directions = new char[]{'N', 'S', 'S', 'E', 'E', 'W'};

for (char c : directions) {
    switch (c) {
        case 'N' -> { y++; }
        case 'S' -> { y--; }
        case 'W' -> { x--; }
        case 'E' -> { x++; }
        default -> { System.out.println("Error! Unknown direction"); }
    }
    System.out.println("The new position is " + x + " , " + y);
}
\end{lstlisting}

Cette syntaxe avec \texttt{->} nécessite Java~14 ou plus récent. Pour les
versions plus anciennes, il faut utiliser \texttt{case} avec \texttt{break}.

\subsection{Chaînes de caractères}

Les variables contenant des chaînes ont le type \texttt{String}. On les
concatène avec \texttt{+}, y compris avec des valeurs primitives :

\begin{lstlisting}[language=Java]
String s1 = "This is a string";
String s2 = "This is another string";
String s3 = s1 + "---" + s2 + 12345;
System.out.println(s3);
\end{lstlisting}

La classe \texttt{String} propose de nombreuses méthodes utiles, par exemple :

\begin{lstlisting}[language=Java]
String s = "Hello world";
int l = s.length();
boolean b = s.isEmpty();
char c = s.charAt(3);
boolean b2 = s.startsWith("Hello");
int i = s.indexOf("wo");
String t = s.substring(2);
\end{lstlisting}

Pour convertir une chaîne en entier :

\begin{lstlisting}[language=Java]
int i = Integer.parseInt("1234");
\end{lstlisting}

\subsection{Modèle mental pour les chaînes}

Comme les tableaux, les variables \texttt{String} contiennent des références.

\subsection{Comparer des valeurs}

Pour les types primitifs, \texttt{==} compare les valeurs. Pour les tableaux et
les chaînes, \texttt{==} compare les références, pas le contenu :

\begin{lstlisting}[language=Java]
int[] a = {1, 2, 3};
int[] b = {1, 2, 3};
System.out.println(a == b); // false

int[] c = a;
System.out.println(a == c); // true
\end{lstlisting}

Pour comparer le contenu de deux chaînes, on utilise \texttt{equals} :

\begin{lstlisting}[language=Java]
String s1 = "Hello" + String.valueOf(1234);
String s2 = "Hello1234";
System.out.println(s1.equals(s2)); // true
\end{lstlisting}

Pour les tableaux, on utilise \texttt{Arrays.equals} de
\texttt{java.util.Arrays}.

\subsection{Énumérations}

Les \texttt{enum} définissent un ensemble de constantes typées. On peut les
utiliser avec \texttt{switch} ou \texttt{==}. Exemple (directions) :

\begin{lstlisting}[language=Java]
public class DirectionFollower {

    public enum Direction {
        NORTH, EAST, SOUTH, WEST;
    }

    public static int[] followDirections(int[] start, Direction[] directions) {
        int[] result = new int[]{start[0], start[1]};

        for (Direction direction : directions) {
            switch (direction) {
                case NORTH: result[1]++; break;
                case EAST:  result[0]++; break;
                case SOUTH: result[1]--; break;
                case WEST:  result[0]--; break;
            }
        }
        return result;
    }

    public static void main(String[] args) {
        int[] start = {0, 0};
        Direction[] directions = {
            Direction.NORTH,
            Direction.EAST,
            Direction.EAST,
            Direction.SOUTH,
            Direction.WEST,
            Direction.NORTH,
            Direction.NORTH
        };

        int[] finalCoordinates = followDirections(start, directions);
        System.out.println(
            "The final coordinates are: [" +
            finalCoordinates[0] + ", " + finalCoordinates[1] + "]"
        );
    }
}
\end{lstlisting}

\section{Exceptions}

En Java, il existe deux façons de sortir d'une méthode : utiliser l'instruction
\texttt{return} ou lever une exception.

\subsection{Lever une exception}

Les exceptions permettent d'interrompre l'exécution normale lorsqu'une situation
exceptionnelle survient. On utilise pour cela l'instruction \texttt{throw} en
lui fournissant une instance de \texttt{Exception} (ou d'une sous-classe) qui
contient des informations sur le problème :

\begin{lstlisting}[language=Java]
class Employee {
    Employee boss;

    void setBoss(Employee boss) throws Exception {
        if (this == boss) {
            throw new Exception("An employee cannot be their own boss");
        } else {
            this.boss = boss;
        }
    }
}
\end{lstlisting}

En général, une méthode susceptible de lever une exception doit l'indiquer dans
sa déclaration avec le mot-clé \texttt{throws}.

Quand une méthode appelle une autre méthode qui peut lever une exception, elle
peut réagir en la \emph{capturant} via un bloc \texttt{try-catch} :

\begin{lstlisting}[language=Java]
public class Main {
    public static void main(String[] args) {

        Employee peter = new Employee();
        Employee anna = new Employee();

        try {
            peter.setBoss(anna);  // OK
            peter.setBoss(peter); // lève une exception
        }
        catch (Exception e) {
            System.out.println("An exception happened: " + e.getMessage());
        }
    }
}
\end{lstlisting}

Si \texttt{setBoss()} lève une exception, l'exécution passe directement dans le
bloc \texttt{catch}. La variable \texttt{e} référence l'objet \texttt{Exception}.

Une méthode peut aussi choisir de ne pas intercepter l'exception : celle-ci est
alors propagée à la méthode appelante, et ainsi de suite, jusqu'à ce qu'elle
soit interceptée :

\begin{lstlisting}[language=Java]
public class Main {
    static void setBossOfTeam(Employee[] team, Employee boss) throws Exception {
        for (Employee employee : team) {
            employee.setBoss(boss); // peut lever une exception
        }
    }

    public static void main(String[] args) {
        Employee peter = new Employee();
        Employee anna = new Employee();

        try {
            Employee[] team = { peter, anna };
            setBossOfTeam(team, peter); // lèvera une exception
        }
        catch (Exception e) {
            System.out.println("An exception happened: " + e.getMessage());
        }
    }
}
\end{lstlisting}

\subsection{Utiliser des sous-classes d'Exception}

En créant des sous-classes de \texttt{Exception}, on fournit plus
d'informations au code qui intercepte l'exception :

\begin{lstlisting}[language=Java]
class SelfBossException extends Exception {
    SelfBossException(String message) {
        super(message);
    }
}

class NoBossException extends Exception {
    NoBossException(String message) {
        super(message);
    }
}

class Employee {
    Employee boss;

    void setBoss(Employee boss) throws SelfBossException, NoBossException {
        if (this == boss) {
            throw new SelfBossException("An employee cannot be their own boss");
        } else if (boss == null) {
            throw new NoBossException("You cannot take the boss away from an employee");
        } else {
            this.boss = boss;
        }
    }
}
\end{lstlisting}

On peut alors écrire :

\begin{lstlisting}[language=Java]
public class Main {
    public static void main(String[] args) {
        Employee peter = new Employee();
        Employee anna = new Employee();

        try {
            peter.setBoss(anna);
            peter.setBoss(null); // lève NoBossException
        }
        catch (SelfBossException e) {
            System.out.println("SelfBossException happened: " + e.getMessage());
        }
        catch (NoBossException e) {
            System.out.println("NoBossException happened: " + e.getMessage());
        }
    }
}
\end{lstlisting}

Ou regrouper :

\begin{lstlisting}[language=Java]
catch (SelfBossException | NoBossException e) {
    System.out.println("Some exception happened: " + e.getMessage());
}
\end{lstlisting}

Ou encore intercepter toutes les exceptions :

\begin{lstlisting}[language=Java]
catch (Exception e) {
    System.out.println("Some exception happened: " + e.getMessage());
}
\end{lstlisting}

\subsection{Exceptions contrôlées et non contrôlées}

Les exceptions que nous avons levées ci-dessus sont \emph{contrôlées}
(\emph{checked}) : le compilateur vérifie qu'elles sont soit interceptées, soit
déclarées avec \texttt{throws}.

Certaines exceptions sont \emph{non contrôlées} (\emph{unchecked}), par
exemple \texttt{NullPointerException}, levée lorsque le programme tente
d'accéder à une référence \texttt{null} :

\begin{lstlisting}[language=Java]
public class Main {
    public static void main(String[] args) {
        Object obj = null;
        String s = obj.toString(); // NullPointerException
    }
}
\end{lstlisting}

Pas besoin de \texttt{throws} ni de \texttt{try-catch}, même si on peut les
intercepter :

\begin{lstlisting}[language=Java]
try {
    String s = obj.toString();
}
catch (NullPointerException e) {
    System.out.println("A null pointer exception happened!");
}
\end{lstlisting}

Les exceptions non contrôlées sont des instances de \texttt{Error} ou de
\texttt{RuntimeException} (ou de leurs sous-classes). Toutes ces classes
dérivent de \texttt{Throwable}.

\subsection{Avons-nous besoin des exceptions ?}

On pourrait réécrire \texttt{setBoss} sans exceptions :

\begin{lstlisting}[language=Java]
class Employee {
    Employee boss;

    boolean setBoss(Employee boss) {
        if (this == boss) {
            return false;
        } else {
            this.boss = boss;
            return true;
        }
    }
}
\end{lstlisting}

Et côté appel :

\begin{lstlisting}[language=Java]
boolean success = peter.setBoss(anna);
if (success) {
    success = peter.setBoss(peter);
}
if (!success) {
    System.out.println("Something bad happened");
}
\end{lstlisting}

Sans exceptions, il faut vérifier le résultat après chaque appel, ce qui alourdit
le code, mais la logique peut parfois être plus explicite. Conclusion :
utilisez les exceptions avec parcimonie.

\subsection{Exceptions et opérations d'entrée/sortie}

Le JDK fournit de nombreuses classes pour travailler avec des fichiers et
communiquer sur Internet. Beaucoup de leurs méthodes peuvent lever
\texttt{IOException}. Exemple : lecture de deux caractères depuis un fichier :

\begin{lstlisting}[language=Java]
import java.io.FileReader;
import java.io.IOException;

public class Main {
    public static void main(String[] args) {
        try {
            FileReader reader = new FileReader("somefile.txt");

            char c1 = (char) reader.read();
            char c2 = (char) reader.read();

            reader.close();
        }
        catch (IOException e) {
            System.out.println(e);
        }
    }
}
\end{lstlisting}

Si \texttt{read()} lève une exception, \texttt{reader.close()} n'est pas
appelé. On corrige avec \texttt{finally} :

\begin{lstlisting}[language=Java]
try {
    FileReader reader = new FileReader("somefile.txt");
    try {
        char c1 = (char) reader.read();
        char c2 = (char) reader.read();
    }
    finally {
        reader.close();
    }
}
catch (IOException e) {
    System.out.println(e);
}
\end{lstlisting}

Java propose une forme compacte, le \emph{try-with-resources} :

\begin{lstlisting}[language=Java]
import java.io.FileReader;
import java.io.IOException;

public class Main {
    public static void main(String[] args) {
        try (FileReader reader = new FileReader("somefile.txt")) {
            char c1 = (char) reader.read();
            char c2 = (char) reader.read();
        }
        catch (IOException e) {
            System.out.println(e);
        }
    }
}
\end{lstlisting}

\chapter{Programmation orientée objet}
\label{chap:oopl}

\section{Notions de base}
\label{sec:oopl-basics}

\subsection{Créer vos propres objets}

Jusqu’ici, nous avons surtout utilisé :

\begin{itemize}
  \item des types primitifs (\texttt{int}, \texttt{double}, \dots),
  \item des tableaux,
  \item la classe \texttt{String}.
\end{itemize}

En Java, nous pouvons définir nos \emph{propres} types d’objets à l’aide de classes.  
Considérons par exemple des employé·e·s décrits par un nom et un salaire :

\begin{lstlisting}[language=Java]
class Employee {
    String name;
    int salary;
}
\end{lstlisting}

Nous pouvons alors créer et manipuler ces objets :

\begin{lstlisting}[language=Java]
public class Main {
    public static void main(String[] args) {
        Employee anna = new Employee();
        anna.name = "Anna";
        anna.salary = 45000;

        Employee peter = new Employee();
        peter.name = "Peter";
        peter.salary = 45000;

        System.out.println(anna.name + " earns " + anna.salary);
        System.out.println(peter.name + " earns " + peter.salary);
    }
}
\end{lstlisting}

Chaque objet possède ses \emph{propres} variables d’instance. Si nous augmentons le salaire d’Anna, cela ne modifie pas celui de Peter, car ce sont deux objets distincts.

\subsection{Initialiser les objets}

Par défaut, en Java :

\begin{itemize}
  \item les champs numériques sont initialisés à \texttt{0},
  \item les références (par exemple \texttt{String}) sont initialisées à \texttt{null}.
\end{itemize}

Pour éviter d’oublier certaines initialisations, on utilise des \emph{constructeurs}. Un constructeur a le même nom que la classe et n’a pas de type de retour.

\begin{lstlisting}[language=Java]
class Employee {
    String name;
    int salary;

    Employee(String name, int salary) {
        this.name = name;
        this.salary = salary;
    }
}
\end{lstlisting}

Le mot-clé \texttt{this} désigne l’objet courant. Il permet aussi de distinguer les paramètres des champs lorsqu’ils portent le même nom :

\begin{lstlisting}[language=Java]
Employee(String name, int salary) {
    // Mauvais :
    // name = name;       // ne fait que réaffecter le paramètre à lui-même

    // Correct :
    this.name = name;
    this.salary = salary;
}
\end{lstlisting}

\subsection{Modèle mental}

Une variable de type classe ne \emph{contient} pas directement l’objet, mais une \emph{référence} vers celui-ci.

\begin{lstlisting}[language=Java]
Employee anna = new Employee("Anna", 45000);
Employee peter = new Employee("Peter", 45000);
\end{lstlisting}

Ici, \texttt{anna} et \texttt{peter} sont deux références distinctes, chacune pointant vers un objet différent en mémoire.

L’opérateur \texttt{==} appliqué à des objets compare les \emph{références}, pas le contenu.

\begin{lstlisting}[language=Java]
Employee x = new Employee("Anna", 45000);
Employee y = new Employee("Anna", 45000);
Employee z = x;

System.out.println(x == y); // false : objets distincts
System.out.println(x == z); // true  : même référence
\end{lstlisting}

\subsection{Travailler avec des objets}

On peut créer des tableaux d’objets :

\begin{lstlisting}[language=Java]
Employee[] staff = new Employee[3];

staff[0] = new Employee("Anna", 45000);
staff[1] = new Employee("Peter", 45000);
staff[2] = new Employee("John", 40000);
\end{lstlisting}

Chaque case du tableau est une \emph{référence} vers un \texttt{Employee}.  
Une référence non initialisée vaut \texttt{null}. Toute tentative d’utiliser une référence \texttt{null} provoque un \texttt{NullPointerException}.

\subsection{Méthodes}

On peut définir des méthodes \emph{statiques} externes :

\begin{lstlisting}[language=Java]
static void increaseSalary(Employee employee, int raise) {
    if (raise < 10000) {
        employee.salary += raise;
    }
}
\end{lstlisting}

Mais en programmation orientée objet, il est préférable que les opérations sur un objet soient définies à \emph{l’intérieur} de sa classe :

\begin{lstlisting}[language=Java]
class Employee {
    String name;
    int salary;

    Employee(String name, int salary) {
        this.name = name;
        this.salary = salary;
    }

    void increaseSalary(int raise) {
        if (raise < 10000) {
            this.salary += raise;
        }
    }
}

public class Main {
    public static void main(String[] args) {
        Employee anna = new Employee("Anna", 45000);
        Employee peter = new Employee("Peter", 45000);

        anna.increaseSalary(2000);
        peter.increaseSalary(3000);

        System.out.println("New salary of Anna is " + anna.salary);
        System.out.println("New salary of Peter is " + peter.salary);
    }
}
\end{lstlisting}

À l’intérieur d’une méthode d’instance, \texttt{this} référence l’objet sur lequel la méthode a été appelée.

\subsection{Restreindre l’accès}

Avec le code précédent, on peut toujours contourner la logique métier :

\begin{lstlisting}[language=Java]
Employee anna = new Employee("Anna", 45000);
anna.salary += 1_500_000; // Oups.
\end{lstlisting}

Pour protéger les données, on utilise le modificateur d’accès \texttt{private} :

\begin{lstlisting}[language=Java]
class Employee {
    String name;
    private int salary;

    Employee(String name, int salary) {
        this.name = name;
        this.salary = salary;
    }

    void increaseSalary(int raise) {
        if (raise < 10000) {
            this.salary += raise;
        }
    }
}
\end{lstlisting}

La variable d’instance \texttt{salary} n’est plus accessible depuis l’extérieur.  
Pour lire sa valeur, on ajoute une méthode d’accès (\emph{getter}) :

\begin{lstlisting}[language=Java]
int getSalary() {
    return this.salary;
}

public class Main {
    public static void main(String[] args) {
        Employee anna = new Employee("Anna", 45000);
        anna.increaseSalary(2000);
        System.out.println("New salary of Anna is " + anna.getSalary());
    }
}
\end{lstlisting}

\subsection{Héritage}

Supposons un jeu où les armes sont représentées par une classe :

\begin{lstlisting}[language=Java]
class Weapon {
    private int level;
    private String name;

    Weapon(String name, int level) {
        this.name = name;
        this.level = level;
    }

    int getPrice() {
        return this.level * 500;
    }

    int getSimpleDamage() {
        return this.level * 10;
    }

    int getDoubleDamage() {
        return this.getSimpleDamage() * 2;
    }
}
\end{lstlisting}

On veut maintenant une \texttt{MightySword} (épée puissante) avec des dégâts différents. On utilise \emph{extends} :

\begin{lstlisting}[language=Java]
class MightySword extends Weapon {

    MightySword(int level) {
        super("Mighty sword", level);
    }

    @Override
    int getSimpleDamage() {
        return super.getSimpleDamage() + 10;
    }
}
\end{lstlisting}

Ici :

\begin{itemize}
  \item \texttt{extends} indique que \texttt{MightySword} \emph{hérite} de \texttt{Weapon},
  \item le constructeur appelle \texttt{super(...)} pour initialiser la super-classe,
  \item \texttt{@Override} signale que l’on redéfinit une méthode existante.
\end{itemize}

\subsection{Le mot-clé \texttt{super}}

\texttt{super} permet :

\begin{itemize}
  \item d’appeler le constructeur de la super-classe,
  \item d’accéder à une méthode ou un champ de la super-classe (\texttt{super.methode()}).
\end{itemize}

Il est utile lorsqu’on spécialise un comportement tout en réutilisant l’implémentation existante.

\subsection{L’annotation \texttt{@Override}}

\texttt{@Override} demande au compilateur de vérifier qu’une méthode redéfinit effectivement une méthode héritée.  
Sans cette annotation, une simple faute de frappe créerait une nouvelle méthode au lieu d’écraser l’ancienne.

\subsection{Étendre encore et encore\ldots{} (polymorphisme)}

On peut enchaîner l’héritage (par exemple différentes sous-classes de \texttt{Weapon}).  
Le \emph{polymorphisme} permet de manipuler des objets de sous-classes via une référence de la super-classe :

\begin{lstlisting}[language=Java]
Weapon w = new MightySword(3);
System.out.println(w.getSimpleDamage()); // Appelle la version de MightySword
\end{lstlisting}

Le choix de la méthode est fait dynamiquement à l’exécution selon le type réel de l’objet.

\subsection{La hiérarchie de classes et \texttt{Object}}

En Java, toute classe hérite (directement ou indirectement) de \texttt{Object}, qui fournit notamment :

\begin{itemize}
  \item \texttt{toString()} : représentation textuelle,
  \item \texttt{equals(Object o)} : test d’égalité logique,
  \item \texttt{hashCode()} : code de hachage.
\end{itemize}

Redéfinir correctement \texttt{toString()}, \texttt{equals()} et \texttt{hashCode()} est essentiel pour les collections.

\subsection{\texttt{ArrayList}}

\texttt{ArrayList<E>} est une implémentation dynamique de liste.

\begin{lstlisting}[language=Java]
import java.util.ArrayList;

ArrayList<String> names = new ArrayList<>();
names.add("Alice");
names.add("Bob");
\end{lstlisting}

On utilise :

\begin{itemize}
  \item \texttt{add}, \texttt{get}, \texttt{size}, \texttt{remove}, \dots
\end{itemize}

\subsection{Boucles \texttt{for} sur \texttt{ArrayList}}

\begin{lstlisting}[language=Java]
// for classique
for (int i = 0; i < names.size(); i++) {
    System.out.println(names.get(i));
}

// for-each
for (String n : names) {
    System.out.println(n);
}
\end{lstlisting}

\subsection{Boxing et unboxing}

Java convertit automatiquement entre types primitifs et leurs classes enveloppes :

\begin{itemize}
  \item \texttt{int} $\leftrightarrow$ \texttt{Integer},
  \item \texttt{double} $\leftrightarrow$ \texttt{Double}, etc.
\end{itemize}

\begin{lstlisting}[language=Java]
ArrayList<Integer> list = new ArrayList<>();
list.add(10);          // boxing automatique
int x = list.get(0);   // unboxing automatique
\end{lstlisting}

\subsection{\texttt{ArrayList} et génériques}

Les collections utilisent des \emph{génériques} pour indiquer le type des éléments :

\begin{lstlisting}[language=Java]
ArrayList<String> list = new ArrayList<>();
list.add("hello");
// list.add(1); // interdit : pas une String
\end{lstlisting}

Cela permet de détecter des erreurs de type à la compilation.

\subsection{Surcharge de méthodes (paramètres différents)}

Deux méthodes dans la même classe peuvent partager le même nom si leurs listes de paramètres sont différentes :

\begin{lstlisting}[language=Java]
void print(String s) { ... }
void print(int x)    { ... }
\end{lstlisting}

Le compilateur choisit la version appropriée selon les arguments.

\subsection{Surcharge avec des types de sous-classes}

Si plusieurs signatures correspondent, Java choisit celle dont les types de paramètres sont les plus spécifiques.

\subsection{Surcharge et ``meilleure correspondance''}

La résolution de surcharge suit des règles précises (conversion, promotion, etc.). Si l’ambiguïté est trop grande, le code ne compile pas.

\subsection{Héritage multiple}

Java \emph{n’autorise pas} l’héritage multiple de classes (une classe ne peut étendre qu’une seule autre classe), afin d’éviter des ambiguïtés.  
La flexibilité supplémentaire se fait via :

\begin{itemize}
  \item les interfaces,
  \item la délégation.
\end{itemize}

\subsection{Le mot-clé \texttt{final}}

\subsubsection{Paramètres \texttt{final}}

Un paramètre déclaré \texttt{final} ne peut plus être réaffecté dans le corps de la méthode.

\subsubsection{Variables locales \texttt{final}}

Même idée : une variable locale \texttt{final} ne peut être réaffectée. Utile pour exprimer qu’une valeur ne doit pas changer.

\subsubsection{Méthodes \texttt{final}}

Une méthode \texttt{final} ne peut pas être redéfinie dans une sous-classe. On garantit ainsi un comportement immuable.

\subsubsection{Classes \texttt{final}}

Une classe \texttt{final} ne peut pas être étendue. Exemple : \texttt{String}. Cela évite qu’une sous-classe ne brise des hypothèses cruciales.

\subsubsection{Variables de classe \texttt{final}}

Typiquement utilisées pour les constantes :

\begin{lstlisting}[language=Java]
class Physics {
    static final double SPEED_OF_LIGHT = 299792458;
}
\end{lstlisting}

\subsubsection{Variables d’instance \texttt{final}}

Une variable d’instance \texttt{final} est fixée une fois pour toutes (souvent dans le constructeur) et ne peut plus être modifiée.  
Pratique pour représenter l’\og identité\fg{} d’un objet (par exemple un numéro de registre national).

\subsection{Paquets (\textit{packages})}
\label{subsec:packages}

Dans des projets plus grands, on place chaque classe dans son propre fichier \texttt{.java} et on les regroupe par \emph{paquets} :

\begin{lstlisting}[language=Java]
// Fichier Person.java
package lepl1402.week3;

class Person {
    final String socialSecurityNumber;

    Person(String ssn) {
        this.socialSecurityNumber = ssn;
    }
}
\end{lstlisting}

\begin{lstlisting}[language=Java]
// Fichier Main.java
package lepl1402.week3;

public class Main {
    public static void main(String[] args) {
        Person person = new Person("123-456-789");
    }
}
\end{lstlisting}

La structure des répertoires doit refléter celle des paquets.

Si aucun \texttt{package} n’est indiqué, la classe appartient au \og package sans nom\fg{}.

\subsection{Utiliser plusieurs paquets}

Les classes de paquets différents ne se voient pas automatiquement. Exemple :

\begin{lstlisting}[language=Java]
// Person.java
package lepl1402.week3.example;

public class Person {
    final String socialSecurityNumber;

    public Person(String ssn) {
        this.socialSecurityNumber = ssn;
    }
}
\end{lstlisting}

\begin{lstlisting}[language=Java]
// Main.java
package lepl1402.week3;

import lepl1402.week3.example.Person;

public class Main {
    public static void main(String[] args) {
        Person person = new Person("123-456-789");
    }
}
\end{lstlisting}

Points importants :

\begin{enumerate}
  \item Seules les classes \texttt{public} sont accessibles depuis d’autres paquets.
  \item Seuls les membres \texttt{public} sont accessibles depuis d’autres paquets.
  \item On utilise \texttt{import} ou le nom pleinement qualifié.
\end{enumerate}

\subsection{Pourquoi les paquets sont-ils utiles ?}

\begin{itemize}
  \item Ils permettent de contrôler la visibilité des classes et méthodes.
  \item Ils offrent des espaces de noms séparés : deux paquets peuvent contenir chacun une classe \texttt{ABC} sans conflit.
\end{itemize}

\subsection{Contrôle d’accès}
\label{subsec:access-control}

Récapitulatif :

\paragraph{Visibilité des classes}

\begin{itemize}
  \item \texttt{public} : visible partout.
  \item sans modificateur : visible uniquement dans le même paquet.
\end{itemize}

\paragraph{Visibilité des membres (champs, méthodes)}

\begin{itemize}
  \item \texttt{public} : accessible partout.
  \item \texttt{private} : accessible uniquement dans la classe.
  \item \texttt{protected} : accessible dans la classe, ses sous-classes et les classes du même paquet.
  \item sans modificateur : accessible dans la classe et dans le même paquet.
\end{itemize}

\section{Classes abstraites}
\label{sec:abstract-classes}

Une classe abstraite ne peut pas être instanciée directement. Elle sert de base commune à plusieurs sous-classes.  
Elle peut contenir :

\begin{itemize}
  \item des méthodes concrètes,
  \item des méthodes abstraites (sans corps).
\end{itemize}

Imaginons un programme de dessin scientifique qui gère différentes formes géométriques. Toutes les formes peuvent :

\begin{itemize}
  \item afficher des informations à leur sujet,
  \item mais la formule de l’aire dépend du type de forme.
\end{itemize}

On applique le principe Open/Closed : ouvert à l’extension, fermé à la modification.  

\begin{lstlisting}[language=Java]
public abstract class Shape {
    protected String shapeName;

    public Shape(String name) {
        this.shapeName = name;
    }

    // Méthode abstraite : chaque sous-classe fournit sa formule
    public abstract double calculateArea();

    // Méthode concrète commune
    public void displayShapeInfo() {
        System.out.println("The " + shapeName +
                           " has an area of: " + calculateArea());
    }
}
\end{lstlisting}

\begin{lstlisting}[language=Java]
public class Circle extends Shape {
    private double radius;

    public Circle(double radius) {
        super("Circle");
        this.radius = radius;
    }

    @Override
    public double calculateArea() {
        return Math.PI * radius * radius;
    }
}

public class Rectangle extends Shape {
    private double length;
    private double width;

    public Rectangle(double length, double width) {
        super("Rectangle");
        this.length = length;
        this.width = width;
    }

    @Override
    public double calculateArea() {
        return length * width;
    }
}

public class Triangle extends Shape {
    private double base;
    private double height;

    public Triangle(double base, double height) {
        super("Triangle");
        this.base = base;
        this.height = height;
    }

    @Override
    public double calculateArea() {
        return 0.5 * base * height;
    }
}
\end{lstlisting}

On peut alors écrire des méthodes génériques :

\begin{lstlisting}[language=Java]
class ShapeUtils {
    public static double calculateTotalArea(Shape[] shapes) {
        double totalArea = 0.0;
        for (Shape shape : shapes) {
            totalArea += shape.calculateArea();
        }
        return totalArea;
    }

    public static void main(String[] args) {
        Shape[] shapes = {
            new Circle(5),
            new Rectangle(4, 5),
            new Triangle(3, 4)
        };
        double totalArea = calculateTotalArea(shapes);
        System.out.println("Total Area: " + totalArea);
    }
}
\end{lstlisting}

L’ajout d’une nouvelle forme ne nécessite pas de modifier \texttt{ShapeUtils} : il suffit de créer une nouvelle sous-classe de \texttt{Shape}.

\section{Interfaces}
\label{sec:interfaces}

Une interface est un type entièrement abstrait : elle spécifie \emph{ce que} doivent faire les classes qui l’implémentent, pas \emph{comment}.

\begin{lstlisting}[language=Java]
public interface Camera {
    void takePhoto();
    void recordVideo();
}

public interface MediaPlayer {
    void playAudio();
    void playVideo();
}
\end{lstlisting}

Une classe peut implémenter plusieurs interfaces (contrairement à l’héritage de classes) :

\begin{lstlisting}[language=Java]
public class Smartphone implements Camera, MediaPlayer {

    @Override
    public void takePhoto() {
        System.out.println("Taking a photo");
    }

    @Override
    public void recordVideo() {
        System.out.println("Recording video");
    }

    @Override
    public void playAudio() {
        System.out.println("Playing audio");
    }

    @Override
    public void playVideo() {
        System.out.println("Playing video");
    }
}
\end{lstlisting}

Les interfaces offrent une grande flexibilité et encouragent la conception modulaire.

\section{Délégation}
\label{sec:delegation}

Considérons une classe \texttt{Book} :

\begin{lstlisting}[language=Java]
public class Book {
    private String title;
    private String author;
    private int publicationYear;

    public Book(String title, String author, int year) {
        this.title = title;
        this.author = author;
        this.publicationYear = year;
    }

    // getters, setters, ...
}
\end{lstlisting}

On veut trier des livres par titre. Une première approche : implémenter \texttt{Comparable<Book>} :

\begin{lstlisting}[language=Java]
public class Book implements Comparable<Book> {
    final String title;
    final String author;
    final int publicationYear;

    public Book(String title, String author, int year) {
        this.title = title;
        this.author = author;
        this.publicationYear = year;
    }

    @Override
    public int compareTo(Book other) {
        return this.title.compareTo(other.title);
    }
}
\end{lstlisting}

\begin{lstlisting}[language=Java]
List<Book> books = new ArrayList<>();
// ... ajout de livres ...
Collections.sort(books); // tri par titre
\end{lstlisting}

Mais si l’on veut trier aussi par auteur ou par année, une unique méthode \texttt{compareTo} ne suffit plus.  
On utilise alors le \emph{design pattern de délégation} avec l’interface \texttt{Comparator} :

\begin{lstlisting}[language=Java]
import java.util.Comparator;

public class TitleComparator implements Comparator<Book> {
    @Override
    public int compare(Book b1, Book b2) {
        return b1.getTitle().compareTo(b2.getTitle());
    }
}

public class AuthorComparator implements Comparator<Book> {
    @Override
    public int compare(Book b1, Book b2) {
        return b1.getAuthor().compareTo(b2.getAuthor());
    }
}

public class YearComparator implements Comparator<Book> {
    @Override
    public int compare(Book b1, Book b2) {
        return Integer.compare(b1.getPublicationYear(),
                               b2.getPublicationYear());
    }
}
\end{lstlisting}

\begin{lstlisting}[language=Java]
Collections.sort(books, new TitleComparator());
Collections.sort(books, new AuthorComparator());
Collections.sort(books, new YearComparator());
\end{lstlisting}

Ici, l’algorithme de tri délègue la comparaison à un objet \texttt{Comparator}. On peut ajouter de nouveaux critères sans modifier \texttt{Book} ni \texttt{Collections.sort} : cela illustre le principe Open/Closed.

\paragraph{Exercice (délégation)}

On développe un système de gestion de documents. La classe \texttt{Document} contient le contenu, mais délègue l’impression à une interface \texttt{Printer} (par exemple \texttt{InkjetPrinter}, \texttt{LaserPrinter}).  
L’objectif est de respecter le principe de responsabilité unique et de faciliter l’extension.

(Énoncé en code identique à la version anglaise, avec les commentaires adaptés.)

\section{Observer}
\label{sec:observer}

Une bonne conception cherche un couplage faible entre objets.  
Le \emph{design pattern Observer} permet à un objet (le sujet) de notifier automatiquement plusieurs observateurs des changements.

\subsection{Pattern Observer sur des composants GUI}

Les bibliothèques Swing/AWT utilisent ce principe. Exemple : un bouton qui affiche ``Thank you'' lorsqu’on clique :

\begin{lstlisting}[language=Java]
import javax.swing.JButton;
import javax.swing.JFrame;
import javax.swing.JOptionPane;
import java.awt.event.ActionEvent;
import java.awt.event.ActionListener;

class ButtonActionListener implements ActionListener {
    @Override
    public void actionPerformed(ActionEvent e) {
        JOptionPane.showMessageDialog(null, "Thank you!");
    }
}

public class AppWithActionListener {
    public static void main(String[] args) {
        JFrame frame = new JFrame("Hello");
        frame.setSize(400, 200);
        frame.setDefaultCloseOperation(JFrame.DISPOSE_ON_CLOSE);

        JButton button = new JButton("Press me!");
        button.addActionListener(new ButtonActionListener());
        frame.add(button);

        frame.setVisible(true);
    }
}
\end{lstlisting}

\begin{itemize}
  \item \texttt{ActionListener} est une interface avec la méthode \texttt{actionPerformed}.
  \item \texttt{ButtonActionListener} l’implémente.
  \item On enregistre un observateur via \texttt{addActionListener}.
\end{itemize}

Le bouton appelle l’observateur lorsqu’un clic survient : \og Don’t call us, we will call you.\fg{}

\subsection{Implémenter le pattern Observer}

Exemple d’un compte bancaire observable :

\begin{lstlisting}[language=Java]
public interface AccountObserver {
    void accountHasChanged(int newValue);
}

class MyObserver implements AccountObserver {
    @Override
    public void accountHasChanged(int newValue) {
        System.out.println("The account has changed. New value: " + newValue);
    }
}
\end{lstlisting}

\begin{lstlisting}[language=Java]
import java.util.LinkedList;
import java.util.List;

public class ObservableAccount {
    private int value;
    private List<AccountObserver> observers = new LinkedList<>();

    public void deposit(int d) {
        value += d;
        for (AccountObserver o : observers) {
            o.accountHasChanged(value);
        }
    }

    public void addObserver(AccountObserver o) {
        observers.add(o);
    }

    public static void main(String[] args) {
        ObservableAccount account = new ObservableAccount();
        MyObserver observerFather = new MyObserver();
        MyObserver observerMother = new MyObserver();
        MyObserver observerGirl = new MyObserver();
        MyObserver observerBoy = new MyObserver();

        account.addObserver(observerFather);
        account.addObserver(observerMother);
        account.addObserver(observerGirl);
        account.addObserver(observerBoy);

        account.deposit(100);
        account.deposit(50);
    }
}
\end{lstlisting}

Remarques :

\begin{itemize}
  \item \texttt{ObservableAccount} est le sujet observé.
  \item Les observateurs sont stockés dans une collection.
  \item À chaque dépôt, tous les observateurs sont notifiés.
\end{itemize}

\paragraph{Exercice (Observer + Swing)}

On vous fournit une application \texttt{MessageApp} avec :

\begin{itemize}
  \item une zone de texte pour saisir un message,
  \item un bouton \og Submit\fg{},
  \item une zone de texte multi-lignes pour afficher les messages.
\end{itemize}

Le message soumis doit passer par un \texttt{SpellChecker}, puis être diffusé à des observateurs via un sujet \texttt{MessageSubject}, et finalement être affiché dans l’UI.

Contraintes :

\begin{itemize}
  \item on doit pouvoir remplacer facilement le \texttt{SpellChecker} sans modifier \texttt{MessageApp},
  \item \texttt{MessageSubject} ne doit pas dépendre directement de \texttt{MessageApp},
  \item privilégier les interfaces pour le couplage.
\end{itemize}

Le code fourni (interfaces \texttt{SpellChecker}, \texttt{MessageObserver}, classe \texttt{MessageSubject}, etc.) doit être complété en appliquant rigoureusement le pattern Observer.

% Fin du chapitre 2
\chapter{Tests unitaires}
\label{chap:unit-testing}

\section{Tests logiciels}
\label{sec:software-testing}

\subsection{Qu'est-ce que le test logiciel ?}
\label{subsec:what-is-testing}

Selon la norme ANSI/IEEE 610.12-1990, le test est « le processus consistant à faire fonctionner un système ou un
composant dans des conditions spécifiées, à observer ou enregistrer les résultats et à évaluer un ou plusieurs aspects
du système ou du composant ». Plus informellement, tester signifie vérifier que le logiciel (ou matériel) fait ce que
nous attendons de lui.

Comme exemple, supposons que nous ayons écrit une méthode qui calcule le quotient $a/b$ de deux nombres
naturels $a$ et $b$ :

\begin{lstlisting}[language=Java]
// returns the quotient a / b
static int division(int a, int b) {
    ...
}
\end{lstlisting}

Nous voulons maintenant savoir si notre implémentation est correcte. Nous l'appelons avec les arguments \texttt{6}
et \texttt{3}, et nous obtenons peut-être \texttt{2}. Cela semble correct. Puis nous appelons la méthode avec
\texttt{12} et \texttt{3}, et nous obtenons \texttt{4} : cela semble toujours bon. Enfin, nous appelons la méthode
avec \texttt{5} et \texttt{2} et nous obtenons \texttt{3}. Est-ce correct, ou attendions-nous \texttt{2} ? Et que
devrait-il se passer si les arguments sont \texttt{4} et \texttt{0} ?

Comme le montre cet exemple, les tests ne sont utiles que si nous avons défini ce que notre programme est censé
faire. Il existe différentes manières de spécifier le comportement attendu :

\begin{enumerate}
  \item Nous pouvons écrire une \emph{spécification formelle}. Dans notre exemple, c'est relativement facile car la
        méthode réalise une opération simple. Une spécification possible serait :
        \[
          \operatorname{division}(a,b) =
          \begin{cases}
            \left\lceil \dfrac{a}{b} \right\rceil & \text{si } b \neq 0,\\[0.4em]
            \text{exception d'erreur} & \text{sinon.}
          \end{cases}
        \]
        D'après cette spécification (remarquez le $\lceil\cdot\rceil$), il est clair que la méthode devrait renvoyer
        \texttt{3} lorsqu'on l'appelle avec \texttt{5} et \texttt{2}, et lever une exception si le second argument vaut
        \texttt{0}.
  \item Pour des programmes plus complexes, difficiles à décrire formellement, la spécification est souvent rédigée
        sous forme textuelle, par des phrases du type : « La méthode renvoie le quotient de deux nombres naturels
        \dots ».
  \item Enfin, lorsqu'on travaille avec des clients, on commence souvent avec une liste d'\emph{exigences
        utilisateur}, c'est-à-dire une description de ce qui est souhaité (« Le programme doit calculer $a/b$ »),
        et c'est à nous d'en déduire une spécification précise.
\end{enumerate}

Les exigences peuvent prendre des formes variées. On distingue en général deux grandes catégories :

\begin{itemize}
  \item \textbf{Exigences fonctionnelles} : décrivent ce que le produit doit être capable de faire.
        \begin{itemize}
          \item « Le programme doit calculer $a/b$. »
          \item « Le programme doit trier une liste. »
          \item « Le programme doit afficher les cinq premiers nombres premiers. »
        \end{itemize}
  \item \textbf{Exigences non fonctionnelles} : décrivent comment le produit doit se comporter.
        \begin{itemize}
          \item « Le programme doit être facile à utiliser. »
          \item « Le programme ne doit pas mettre plus de 10 millisecondes pour calculer $a/b$. »
          \item « Le programme doit être sécurisé. »
        \end{itemize}
\end{itemize}

\subsection{Que peut-on tester ?}
\label{subsec:what-can-be-tested}

Les tests peuvent être effectués à différents niveaux.

Dans l'exemple précédent, le test de la méthode \texttt{division()} est un \emph{test unitaire}, car les méthodes
sont les plus petites entités en Java que l'on puisse tester indépendamment.

On peut aussi tester :

\begin{itemize}
  \item une classe ou un paquet complet : \emph{test de module},
  \item plusieurs paquets ensemble : \emph{test d'intégration},
  \item un programme complet : \emph{test système}.
\end{itemize}

Dans ce livre, nous nous concentrerons sur les tests unitaires.

Plus un programme est gros et complexe, plus les tests sont compliqués et, surtout, plus la correction d'un bug peut
être coûteuse. Un test unitaire peut être réalisé dès l'écriture d'une méthode ; si l'on trouve un bug, il est encore
relativement facile de corriger l'implémentation. Imaginez l'inverse : terminer tout le programme, et découvrir
plusieurs mois plus tard, lors d'un test système, que le programme viole la spécification ; dans le pire des cas, il
faudrait tout réécrire.

Il est donc judicieux de commencer à tester le plus tôt possible à l'aide de tests unitaires. Attention toutefois : les
tests unitaires ne remplacent pas les autres niveaux de tests, car certains problèmes n'apparaissent que lorsque
plusieurs méthodes ou classes interagissent.

\subsection{Comment planifier un test unitaire}
\label{subsec:plan-unit-test}

Nous testons un logiciel afin de vérifier qu'il produit les bons résultats (exigences fonctionnelles) de la bonne manière
(exigences non fonctionnelles).

En général, un programme comme notre méthode \texttt{division()} prend une ou plusieurs valeurs en entrée et
produit une sortie. L'ensemble des valeurs d'entrée possibles s'appelle le \emph{domaine d'entrée}. Tester la méthode
avec toutes les valeurs possibles serait l'idéal, mais pratiquement impossible.

Cela nous amène à une approche plus intelligente :

\begin{enumerate}
  \item \textbf{Déterminer le domaine d'entrée.}  
        Ici, la méthode a deux paramètres entiers ; le domaine d'entrée est donc
        $\{-2^{31}..\ 2^{31}-1\} \times \{-2^{31}..\ 2^{31}-1\}$.
  \item \textbf{Découper le domaine en sous-domaines intéressants.}  
        Par exemple, il est intéressant de tester le cas \texttt{b == 0}.  
        On obtient :
        \begin{itemize}
          \item $b = 0$ :
                $\{-2^{31}..\ 2^{31}-1\} \times \{0\}$,
          \item $b \neq 0$ :
                $\{-2^{31}..\ 2^{31}-1\} \times
                 \big(\{-2^{31}..\ 2^{31}-1\} \setminus \{0\}\big)$.
        \end{itemize}
  \item \textbf{Choisir quelques valeurs représentatives par sous-domaine.}  
        Par exemple :
        \begin{itemize}
          \item $a=3, b=0$ pour le cas interdit,
          \item $a=6, b=3$ pour le cas autorisé.
        \end{itemize}
\end{enumerate}

Cette approche fonctionne aussi pour des méthodes plus complexes. Considérons :

\begin{lstlisting}[language=Java]
// returns a sorted array with the elements
// of "a" in ascending order
static int[] sortArray(int[] a) {
    ...
}
\end{lstlisting}

Le domaine d'entrée contient tous les tableaux d'entiers de longueur $n \ge 0$, avec toutes les valeurs possibles.
Des sous-domaines pertinents pourraient être :

\begin{enumerate}
  \item tableau vide ($n = 0$),
  \item tableau avec un seul élément ($n = 1$),
  \item tableaux aléatoires non triés ($n > 1$),
  \item tableaux déjà triés par ordre croissant ($n > 1$),
  \item tableaux triés par ordre décroissant ($n > 1$).
\end{enumerate}

Il est souhaitable que les sous-domaines soient disjoints et couvrent l'ensemble du domaine d'entrée. Si, par exemple,
le cas $n=1$ est déjà couvert dans un sous-domaine, il n'est pas nécessaire de le répéter ailleurs.

\section{Couverture de tests}
\label{sec:test-coverage}

\subsection{Tests en boîte noire et en boîte blanche}
\label{subsec:black-white-box}

Comme le domaine d'entrée d'un programme non trivial est immense, choisir de bons cas de test est un défi majeur.

\begin{itemize}
  \item Si nous n'avons pas accès au code source, nous choisissons les valeurs de test à partir de la spécification et de
        notre expérience : c'est le \emph{test en boîte noire}.
  \item Si nous avons accès au code source (ce sera notre hypothèse), nous pouvons l'utiliser pour choisir des tests
        pertinents : c'est le \emph{test en boîte blanche}.
\end{itemize}

\subsection{Graphe de flot de contrôle et couverture de nœuds}
\label{subsec:cfg-node-coverage}

Considérons une implémentation fautive de \texttt{min()} :

\begin{lstlisting}[language=Java]
// returns the minimum of "a" and "b"
static int min(int a, int b) {
    int m;
    if (a < b)
        m = a;
    else
        m = a; // oops, cela devrait être "m = b"
    return m;
}
\end{lstlisting}

Si nous appelons cette méthode avec des nombres \texttt{a} et \texttt{b} tels que \texttt{a < b}
(par exemple \texttt{3} et \texttt{5}), le bug ne sera jamais exécuté, et le résultat sera toujours correct. On ne peut
découvrir un bug que si le flux d'exécution atteint l'instruction fautive.

On représente souvent le comportement avec un \emph{graphe de flot de contrôle} (CFG), où :

\begin{itemize}
  \item les nœuds représentent des blocs d'instructions,
  \item les arêtes représentent les transitions possibles entre ces blocs.
\end{itemize}

La \emph{couverture de nœuds} consiste à choisir des tests de sorte que chaque nœud du CFG soit exécuté au moins
une fois. C'est un objectif important, mais pas suffisant.

\subsection{Couverture des arêtes}
\label{subsec:edge-coverage}

Même avec 100\% de couverture de nœuds, certains bugs peuvent passer inaperçus. Prenons :

\begin{lstlisting}[language=Java]
static int min(int a, int b) {
    int m = 0;
    if (a < b) {
        m = a;
    }
    // oops : on a oublié le "else" pour le cas a >= b
    return m;
}
\end{lstlisting}

Si nous testons seulement avec \texttt{a = 3, b = 5}, le flux suit toujours le même chemin, et nous obtenons
le bon résultat. Pourtant, le code est incorrect pour d'autres valeurs.

La \emph{couverture des arêtes} exige que chaque arête (transition) du CFG soit empruntée au moins une fois. Elle est
plus stricte que la couverture de nœuds, mais ne garantit toujours pas l'absence de bugs.

\subsection{Couverture des chemins}
\label{subsec:path-coverage}

Considérons maintenant une méthode avec une boucle :

\begin{lstlisting}[language=Java]
// returns the sum of the values 1 to n
static int sum(int n) {
    int sum = 0;
    int i = 1;
    while (i <= n) {
        sum = i;   // oops, devrait être: sum += i;
        i++;
    }
    return sum;
}
\end{lstlisting}

\begin{itemize}
  \item Pour \texttt{n = 0}, la boucle n'est jamais exécutée, le résultat est \texttt{0} : ce test ne révèle rien.
  \item Pour \texttt{n = 1}, la boucle s'exécute une fois, le résultat est \texttt{1} : encore correct \emph{par hasard}.
\end{itemize}

Avec ces deux tests, toutes les arêtes du CFG sont couvertes, mais nous n'avons toujours pas trouvé le bug.  
L'idée de \emph{couverture des chemins} est de considérer des séquences complètes d'arêtes. Malheureusement, dès
qu'il y a des boucles, il existe théoriquement une infinité de chemins possibles : viser 100\% de couverture des chemins
n'est pas réaliste. On se contente donc de choisir des chemins représentatifs (par exemple, zéro itération, une
itération, plusieurs itérations).

\subsection{Chemins « cachés »}
\label{subsec:hidden-paths}

Certains chemins d'exécution ne sont pas visibles directement dans le code source. Par exemple :

\begin{lstlisting}[language=Java]
int r = a / b;
\end{lstlisting}

En Java, cette instruction peut lever une \texttt{ArithmeticException} si \texttt{b == 0}. On peut donc la voir
comme équivalente à :

\begin{lstlisting}[language=Java]
if (b == 0)
    throw new ArithmeticException();
else
    r = a / b;
\end{lstlisting}

Si votre objectif est une couverture « complète », il faut aussi prévoir un cas de test avec \texttt{b == 0}.

\subsection{Outils de mesure de couverture}
\label{subsec:coverage-tools}

\texttt{JaCoCo} est un outil (et une bibliothèque) pour mesurer la couverture des tests de programmes Java.
Lorsqu'on exécute un programme avec JaCoCo, il calcule notamment :

\begin{itemize}
  \item la \textbf{couverture des instructions bytecode} JVM (analogue à la couverture de nœuds, mais au niveau
        bytecode plutôt que des instructions Java),
  \item la \textbf{couverture des branches}, qui correspond à la couverture des différents résultats possibles des
        instructions conditionnelles (\texttt{if}, \texttt{switch}, etc.).
\end{itemize}

Des outils similaires existent pour d'autres langages. Ils aident à vérifier si l'on a écrit suffisamment de cas de test.

\section{Tests unitaires automatisés}
\label{sec:automated-unit-testing}

\subsection{Écrire les tests comme un programme}
\label{subsec:tests-as-programs}

Les tests sont des tâches répétitives. En test unitaire, nous devons tester chaque nouvelle méthode que nous écrivons,
et répéter ces tests à chaque modification du code. Il est donc naturel de laisser l'ordinateur exécuter ces tests pour
nous.

Reprenons la méthode \texttt{min()} :

\begin{lstlisting}[language=Java]
class Main {
    static int min(int a, int b) {
        ...
    }
}
\end{lstlisting}

Nous pouvons écrire un petit programme de test :

\begin{lstlisting}[language=Java]
// test case 1
int result1 = Main.min(3, 5);
if (result1 != 3) {
    System.out.println("Test 1 failed: Minimum of 3 and 5 should be 3");
}

// test case 2
int result2 = Main.min(5, 3);
if (result2 != 3) {
    System.out.println("Test 2 failed: Minimum of 5 and 3 should be 3");
}
\end{lstlisting}

Chaque combinaison de valeurs d'entrée et de résultat attendu est un \emph{cas de test}.

L'avantage d'un tel programme est que nous pouvons relancer facilement tous les tests après chaque changement de
code. Certains développeurs vont plus loin et écrivent d'abord les tests, puis le code qui les fait passer : c'est le
\emph{Test Driven Development} (TDD).

\subsection{JUnit}
\label{subsec:junit}

En Java, la bibliothèque la plus connue pour écrire des tests unitaires automatisés est \texttt{JUnit}. Des outils
similaires existent pour d'autres langages.

Pour écrire des tests avec JUnit, on crée en général une nouvelle classe (par exemple \texttt{MainTest}) et une
méthode par cas de test.

Avec JUnit~4, nos deux tests pour \texttt{min()} peuvent s'écrire :

\begin{lstlisting}[language=Java]
import static org.junit.Assert.* ;

public class MainTest {

    @org.junit.Test
    public void testFirstNumberLessThanSecondNumber() {
        assertEquals("Minimum of 3 and 5 should be 3",
                     3, Main.min(3, 5));
    }

    @org.junit.Test
    public void testFirstNumberGreaterThanSecondNumber() {
        assertEquals("Minimum of 5 and 3 should be 3",
                     3, Main.min(5, 3));
    }
}
\end{lstlisting}

Avec JUnit~5, la syntaxe est légèrement différente :

\begin{lstlisting}[language=Java]
import org.junit.jupiter.api.Assertions;
import org.junit.jupiter.api.Test;

public class MainTest {

    @Test
    public void testFirstNumberLessThanSecondNumber() {
        Assertions.assertEquals(3, Main.min(3, 5),
                "Minimum of 3 and 5 should be 3");
    }

    @Test
    public void testFirstNumberGreaterThanSecondNumber() {
        Assertions.assertEquals(3, Main.min(5, 3),
                "Minimum of 5 and 3 should be 3");
    }
}
\end{lstlisting}

La méthode \texttt{assertEquals} compare la valeur attendue avec la valeur produite par l'implémentation. Un message
(d'usage facultatif) peut être fourni ; il sera affiché si le test échoue.

L'annotation \texttt{@Test} indique à JUnit quelles méthodes exécuter comme tests. IntelliJ utilise également ces
annotations pour afficher les petits triangles verts permettant de lancer un test individuel ou l'ensemble des tests.

La classe \texttt{Assertions} fournit de nombreuses autres méthodes utiles, par exemple :

\begin{itemize}
  \item \texttt{assertArrayEquals()} pour comparer des tableaux,
  \item \texttt{assertNotEquals()} pour vérifier une inégalité,
  \item \texttt{assertSame()} pour vérifier que deux références pointent le même objet.
\end{itemize}

Attention : pour les objets, \texttt{assertEquals()} utilise la méthode \texttt{equals()}, pas \texttt{==}. Pour tester
l'égalité de références, il faut utiliser \texttt{assertSame()}.

\subsection{Aspects pratiques des tests unitaires}
\label{subsec:practical-unit-testing}

L'idée des tests unitaires est d'organiser le programme en petites unités testables. En Java, les méthodes jouent ce
rôle. Si une méthode est trop complexe ou fait plusieurs choses différentes, elle devient plus difficile à tester.

Considérons le code (incomplet) suivant :

\begin{lstlisting}[language=Java]
class DifficultToTest {
    static int m(int v1) {
        // quelque chose de complexe utilisant v1 pour calculer v2
        int v2 = ...;
        // quelque chose de complexe utilisant v2 pour calculer le résultat
        int result = ...;
        return result;
    }
}
\end{lstlisting}

En tant que développeur, nous aimerions vérifier séparément si \texttt{v2} et le résultat final sont correctement
calculés. Il est plus simple de découper la méthode :

\begin{lstlisting}[language=Java]
class EasierToTest {

    static int m1(int v1) {
        // calcule v2 à partir de v1
        int v2 = ...;
        return v2;
    }

    static int m2(int v2) {
        // calcule le résultat à partir de v2
        int result = ...;
        return result;
    }

    static int m(int v1) {
        int v2 = m1(v1);
        int result = m2(v2);
        return result;
    }
}
\end{lstlisting}

Ce nouveau code est plus lisible et plus facile à tester : on peut fournir des valeurs choisies pour \texttt{v1} et
\texttt{v2} afin de tester les deux parties indépendamment.

Un autre aspect pratique concerne les méthodes non statiques ou dépendant d'objets. Par exemple :

\begin{lstlisting}[language=Java]
class Employee {
    private int salary;

    public Employee(int s) {
        salary = s;
    }

    public void increaseSalary(int s) {
        salary += s;
    }

    public int getSalary() {
        return salary;
    }
}
\end{lstlisting}

Pour tester \texttt{increaseSalary()}, il faut d'abord créer un objet :

\begin{lstlisting}[language=Java]
import org.junit.jupiter.api.Assertions;
import org.junit.jupiter.api.Test;

public class EmployeeTest {

    @Test
    void testSalaryIncrease() {
        Employee employee = new Employee(1000);

        Assertions.assertEquals(1000, employee.getSalary());

        employee.increaseSalary(500);

        Assertions.assertEquals(1500, employee.getSalary());
    }
}
\end{lstlisting}

Certains développeurs préfèrent des méthodes de test très petites avec un seul \texttt{assert} chacune ; d'autres
acceptent plusieurs assertions par test lorsqu'elles sont liées. Il s'agit en grande partie d'une question de style : à vous
de choisir une approche claire et cohérente.

% Fin du chapitre 3
\chapter{Algorithmes et structures de données}
\label{chap:algo-ds}

\section{Complexité temporelle}
\label{sec:time-complexity}

Dans le monde en constante évolution de l'informatique, l'efficacité d'un algorithme est primordiale. À mesure que nous cherchons à résoudre des problèmes de plus en plus complexes et à manipuler des volumes de données toujours plus importants, comprendre les performances de nos algorithmes devient plus important que jamais.

C'est là qu'intervient la notion de \emph{complexité temporelle}.

La complexité temporelle fournit une estimation \emph{théorique} du temps nécessaire à l'exécution d'un algorithme en fonction de la taille de ses données d'entrée. Autrement dit, elle nous permet de \emph{prédire} l'efficacité de notre code avant même de l'exécuter — comme une boule de cristal nous indiquant comment l'algorithme se comportera ``dans la nature'' !

Pour en dévoiler les rouages, commençons par une méthode très simple \texttt{sum()} qui calcule la somme de tous les éléments d'un tableau d'entiers passé en argument :

\begin{lstlisting}[language=Java]
// Sums all elements of the array
public class Main {
    public static int sum(int[] values) {
        int total = 0;
        for (int i = 0; i < values.length; i++) {
            total += values[i];
        }
        return total;
    }
}
\end{lstlisting}

On peut mesurer le temps d'exécution à l'aide de la méthode \texttt{System.currentTimeMillis()}, qui renvoie le temps courant en millisecondes depuis l'\emph{Unix Epoch} (1er janvier 1970, 00:00:00 UTC). Voici un exemple qui mesure la durée d'un appel à \texttt{sum()} :

\begin{lstlisting}[language=Java]
public class Main {
    public static void main(String[] args) {
        int[] values = {1, 2, 3, 4, 5};
        long startTime = System.currentTimeMillis();
        int totalSum = sum(values);
        long endTime   = System.currentTimeMillis();
        long duration  = (endTime - startTime); // durée en millisecondes
        System.out.println("sum = " + totalSum + " in " + duration + " ms");
    }
}
\end{lstlisting}

Si l'on fait varier la taille du tableau \texttt{values}, on peut observer l'évolution du temps d'exécution en fonction de la taille de l'entrée et la tracer. Voici ce que l'on obtient sur un ordinateur portable standard :

\begin{figure}[h]
  \centering
  \includegraphics[width=.85\linewidth]{chap4_imgs/chap4_p88_img1.png}
  \caption{Évolution des mesures de temps pour \texttt{sum} sur des tableaux de taille croissante.}
  \label{fig:sum-scaling}
\end{figure}

Sans surprise, le temps absolu dépend fortement de la machine utilisée. Le \emph{même} code, sur un autre ordinateur, produira des mesures différentes. Néanmoins, on remarque que l'évolution du temps est \emph{linéaire} par rapport à la taille du tableau (cf. la droite de tendance).

Question cruciale : pouvait-on le prévoir sans exécuter le code ? La réponse est \emph{oui}. Les travaux fondateurs d'Hartmanis et Stearns permettent d'analyser théoriquement un algorithme à partir de son (pseudo-)code, sans mesures d'exécution. Pour cela, il nous faut d'abord un modèle de calcul simple.

\subsection{Le modèle RAM (\textit{Random Access Machine})}
\label{subsec:ram-model}

Le modèle RAM est une abstraction mathématique d'un ordinateur pour l'analyse d'algorithmes. Dans ce modèle, chaque opération ``simple'' (addition, soustraction, multiplication, division, comparaison, opérations binaires, suivi d'une référence, adressage direct en mémoire, etc.) prend un \emph{temps constant}. On suppose en particulier que l'accès mémoire (par exemple \texttt{values[i]}) s'effectue en temps constant, indépendamment de l'adresse. C'est l'idée de ``\emph{random access}'' : tout emplacement mémoire est accessible en le même temps.

Cette abstraction est raisonnable pour de nombreux cas pratiques (de la même façon que les lois de Newton constituent une bonne approximation de la relativité générale).

\subsection{Meilleur cas et pire cas d'un algorithme}
\label{subsec:best-worst}

Des entrées de \emph{même taille} peuvent forcer un algorithme à effectuer plus ou moins d'étapes. Illustrons-le avec une recherche linéaire qui renvoie l'indice de la première occurrence d'une valeur cible (ou \texttt{-1} si elle est absente) :

\begin{lstlisting}[language=Java]
/**
 * Effectue une recherche linéaire dans un tableau.
 *
 * @param arr le tableau d'entrée
 * @param x   la valeur recherchée
 * @return l'indice de la première occurrence de x, ou -1 si absente
 */
public static int linearSearch(int[] arr, int x) {
    for (int i = 0; i < arr.length; i++) {
        if (arr[i] == x) {
            return i;
        }
    }
    return -1;
}
\end{lstlisting}

Ici, le nombre d'étapes dépend fortement de la \emph{position} de la valeur cible : si elle est au début, l'algorithme termine très vite (\emph{meilleur cas}); si elle est à la fin (ou absente), il parcourt tout le tableau (\emph{pire cas}). Pour certaines familles d'algorithmes, le temps dépend surtout de la \emph{taille} de l'entrée, plus que de son \emph{contenu} (comme pour \texttt{sum()} ci-dessus).

La notation introduite ci-dessous nous permettra de caractériser rigoureusement ces scénarios.

\subsection{Les classes \(\mathcal{O}\), \(\Omega\) et \(\Theta\)}
\label{subsec:big-oh}

Nous nous intéressons à la façon dont le temps d'exécution évolue quand la taille d'entrée \(n\) devient grande. Les \emph{notations asymptotiques} suivantes décrivent des \emph{classes de fonctions} et abstraient les constantes et les termes de plus bas degré :

\[
\begin{aligned}
&f(n) \in \mathcal{O}(g(n))
&&\Longleftrightarrow&&
\exists\, c \in \mathbb{R}^+,\ \exists\, n_0 \in \mathbb{N}:\ \forall n \ge n_0,\ f(n) \le c \cdot g(n)
&&\text{(borne \emph{supérieure})},\\[0.3em]
&f(n) \in \Omega(g(n))
&&\Longleftrightarrow&&
\exists\, c \in \mathbb{R}^+,\ \exists\, n_0 \in \mathbb{N}:\ \forall n \ge n_0,\ f(n) \ge c \cdot g(n)
&&\text{(borne \emph{inférieure})},\\[0.3em]
&f(n) \in \Theta(g(n))
&&\Longleftrightarrow&&
\exists\, c_1,c_2 \in \mathbb{R}^+,\ \exists\, n_0 \in \mathbb{N}:\ \forall n \ge n_0,\
c_1 g(n) \le f(n) \le c_2 g(n)
&&\text{(borne \emph{exacte})}.
\end{aligned}
\]

Intuition : on ne s'intéresse pas aux petites constantes, mais au \emph{comportement dominant} quand \(n\) devient grand (par ex. \(10^4\), \(10^6\)). Cela autorise des simplifications : si
\(f(n) = c \cdot n^a + d \cdot n^b\) avec \(a \ge b \ge 0\) et \(c,d \ge 0\), alors \(f(n) \in \Theta(n^a)\), même si \(c\) est petit et \(d\) très grand. Autrement dit, on ne garde que le \emph{terme de plus haut degré} et on \emph{omet les constantes}.

\subsection{Exemples pratiques d'algorithmes}
\label{subsec:examples}

Quelques classes de complexité courantes et exemples associés :

\begin{center}
\begin{tabular}{ll}
\toprule
\textbf{Complexité} & \textbf{Exemple d'algorithme}\\
\midrule
\(\mathcal{O}(1)\) (constante) & Somme de deux entiers\\
\(\mathcal{O}(\log n)\) (logarithmique) & Recherche dans un tableau trié (recherche binaire)\\
\(\mathcal{O}(n)\) (linéaire) & Somme des éléments ou recherche dans un tableau non trié\\
\(\mathcal{O}(n \log n)\) (linéarithmique) & Tri efficace d'un tableau (tri fusion)\\
\(\mathcal{O}(n^2)\) (quadratique) & Tri par insertion (inefficace)\\
\(\mathcal{O}(n^3)\) (cubique) & Énumération de tous les triplets d'un tableau\\
\(\mathcal{O}(2^n)\) (exponentielle) & Sous-ensemble à somme nulle (\textit{subset-sum})\\
\(\mathcal{O}(n!)\) (factorielle) & TSP : tournée minimale visitant toutes les villes\\
\bottomrule
\end{tabular}
\end{center}

\paragraph{Recherche binaire.} La recherche binaire (dichotomique) trouve la position d'une valeur cible dans un tableau \emph{trié}, en divisant l'intervalle de recherche par deux à chaque étape :

\begin{enumerate}
  \item on examine l'élément du \emph{milieu} ;
  \item s'il est égal à la cible : trouvé ;
  \item s'il est plus grand : on continue dans la moitié \emph{gauche} ;
  \item s'il est plus petit : on continue dans la moitié \emph{droite} ;
  \item on répète jusqu'à trouver ou épuiser l'intervalle.
\end{enumerate}

Sur un tableau de \(16\) valeurs, le pire cas requiert au plus \(4\) essais (\(\log_2 16 = 4\)).

\begin{lstlisting}[language=Java]
public static int binarySearch(int[] arr, int x) {
    int left = 0, right = arr.length - 1;
    while (left <= right) {
        int mid = left + (right - left) / 2; // éviter l'overflow
        if (arr[mid] == x) return mid;
        if (arr[mid] <  x) left  = mid + 1;   // ignorer la moitié gauche
        else               right = mid - 1;   // ignorer la moitié droite
    }
    return -1; // absent
}
\end{lstlisting}

\textbf{Astuce :} \(\texttt{left + (right - left) / 2}\) est préférable à \(\texttt{(left + right) / 2}\) pour éviter un dépassement d'entier (\emph{overflow}) quand \texttt{left + right} dépasse \(2^{31}-1\). Avec des \emph{objets}, on utilisera \texttt{equals()} au lieu de \texttt{==}.

\begin{figure}[h]
  \centering
  \includegraphics[width=.85\linewidth]{chap4_imgs/chap4_p90_img1.png}
  \caption{Illustration pas à pas de la recherche binaire (\(n=16\)).}
  \label{fig:binsearch-1}
\end{figure}

\begin{figure}[h]
  \centering
  \includegraphics[width=.85\linewidth]{chap4_imgs/chap4_p91_img1.png}
  \caption{Suite de l'illustration de la recherche binaire.}
  \label{fig:binsearch-2}
\end{figure}

\paragraph{Exemple \(\mathcal{O}(\log n)\) : nombre de bits.} Le nombre de bits nécessaires pour représenter un entier positif est \(\Theta(\log n)\). Exemple :

\begin{lstlisting}[language=Java]
// Renvoie le nombre de bits nécessaires pour représenter n (>0)
public static int bitCount(int n) {
    int bitCount = 0;
    while (n > 0) {
        bitCount++;
        n = n >> 1; // décalage à droite (division par 2)
    }
    return bitCount;
}
\end{lstlisting}

\paragraph{Remarques sur les cas meilleur/pire.}
La recherche binaire est en \(\Omega(1)\) (meilleur cas : trouvé du premier coup) et en \(\mathcal{O}(\log n)\) (pire cas : on divise l'intervalle jusqu'à 1).

\bigskip
\noindent
Les autres classes citées ci-dessus seront illustrées par les sections suivantes (tri, structures de données, itérateurs, etc.).

% Suite du chapitre 4 : compléments de la section 4.1 puis sections 4.2 à 4.5

\paragraph{Recherche linéaire (\(\mathcal{O}(n)\)).}

Nous avons déjà vu l’algorithme \texttt{sum} et sa complexité \(\Theta(n)\). Un autre exemple typique de complexité
linéaire est la recherche séquentielle :

\begin{lstlisting}[language=Java]
/**
 * Recherche linéaire d'une valeur dans un tableau.
 *
 * @param arr le tableau d'entrée
 * @param x   la valeur recherchée
 * @return l'indice de la première occurrence de x, ou -1 si absente
 */
public static int linearSearch(int[] arr, int x) {
    for (int i = 0; i < arr.length; i++) {
        if (arr[i] == x) {
            return i;
        }
    }
    return -1;
}
\end{lstlisting}

Dans le pire cas (valeur absente ou en dernière position), on parcourt tout le tableau : complexité \(\mathcal{O}(n)\).  
Dans le meilleur cas (première case), un seul test suffit : complexité \(\Omega(1)\).

\paragraph{Tri fusion (\emph{Merge sort}, \(\mathcal{O}(n \log n)\)).}

Le tri fusion est un algorithme « diviser pour régner » :

\begin{itemize}
  \item si le tableau est de taille 0 ou 1 : déjà trié ;
  \item sinon : on le découpe en deux sous-tableaux, on trie récursivement chacun d’eux, puis on fusionne.
\end{itemize}

\begin{lstlisting}[language=Java]
public class MergeSort {

    private static void merge(int[] left, int[] right, int[] result) {
        int i = 0; // index dans left
        int j = 0; // index dans right
        int k = 0; // index dans result

        while (i < left.length && j < right.length) {
            if (left[i] <= right[j]) {
                result[k++] = left[i++];
            } else {
                result[k++] = right[j++];
            }
        }
        while (i < left.length) {
            result[k++] = left[i++];
        }
        while (j < right.length) {
            result[k++] = right[j++];
        }
    }

    public static void mergeSort(int[] values) {
        if (values.length <= 1) { // tableau de taille 0 ou 1 : déjà trié
            return;
        }

        int mid = values.length / 2;
        int[] left  = new int[mid];
        int[] right = new int[values.length - mid];

        System.arraycopy(values, 0,        left,  0,   mid);
        System.arraycopy(values, mid,      right, 0,   values.length - mid);

        mergeSort(left);
        mergeSort(right);

        merge(left, right, values);
    }
}
\end{lstlisting}

Il y a \(\Theta(\log n)\) niveaux de découpe/fusion et chaque niveau traite \(\Theta(n)\) éléments : complexité
\(\Theta(n \log n)\).

\begin{figure}[h]
  \centering
  \includegraphics[width=.7\linewidth]{chap4_imgs/chap4_p94_img1.png}
  \caption{Illustration schématique du tri fusion.}
\end{figure}

\paragraph{Tri par insertion (\(\mathcal{O}(n^2)\)).}

Le tri par insertion insère chaque élément à sa place dans la partie déjà triée à gauche :

\begin{lstlisting}[language=Java]
/**
 * Trie le tableau en place par insertion.
 */
public static void insertionSort(int[] arr) {
    for (int i = 1; i < arr.length; i++) {
        int key = arr[i];
        int j = i - 1;

        // Décale vers la droite les éléments > key
        while (j >= 0 && arr[j] > key) {
            arr[j + 1] = arr[j];
            j--;
        }
        arr[j + 1] = key;
    }
}
\end{lstlisting}

Dans le pire cas, pour chaque élément on décale presque toute la partie gauche : \(\mathcal{O}(n^2)\).  
Dans le meilleur cas (tableau déjà trié), on ne décale jamais : \(\Omega(n)\).

\paragraph{Énumération de tous les triplets (\(\mathcal{O}(n^3)\)).}

On teste si trois éléments quelconques d’un tableau somment à zéro :

\begin{lstlisting}[language=Java]
/**
 * Vérifie s'il existe un triplet d'éléments dont la somme vaut 0.
 */
public static boolean checkTripleSum(int[] arr) {
    int n = arr.length;
    for (int i = 0; i < n - 2; i++) {
        for (int j = i + 1; j < n - 1; j++) {
            for (int k = j + 1; k < n; k++) {
                if (arr[i] + arr[j] + arr[k] == 0) {
                    return true;
                }
            }
        }
    }
    return false;
}
\end{lstlisting}

Il y a \(\frac{n(n-1)(n-2)}{6} = \Theta(n^3)\) triplets possibles : pire cas \(\mathcal{O}(n^3)\).  
Meilleur cas \(\Omega(1)\) si le premier triplet testé convient.

\paragraph{Problème du sous-ensemble de somme nulle (Subset-Sum, \(\mathcal{O}(2^n)\)).}

On généralise en cherchant \emph{un sous-ensemble quelconque} de valeurs dont la somme vaut zéro.  
Algorithme naïf : on énumère récursivement tous les sous-ensembles.

\begin{lstlisting}[language=Java]
/**
 * Vérifie s'il existe un sous-ensemble de arr qui somme à 0.
 */
public static boolean hasZeroSumSubset(int[] arr) {
    return isSubsetSum(arr, 0, 0);
}

private static boolean isSubsetSum(int[] arr, int i, int sum) {
    if (i == arr.length) {
        return sum == 0;
    }
    // cas 1 : on n'inclut pas arr[i]
    if (isSubsetSum(arr, i + 1, sum)) {
        return true;
    }
    // cas 2 : on inclut arr[i]
    return isSubsetSum(arr, i + 1, sum + arr[i]);
}
\end{lstlisting}

Il y a \(2^n\) sous-ensembles possibles : complexité dans le pire cas \(\mathcal{O}(2^n)\).  
Meilleur cas \(\Omega(1)\) si une solution est trouvée immédiatement.

Ce problème est \emph{NP-complet} : aucune solution polynomiale n’est connue (ni exclue). Des algorithmes plus
sophistiqués existent (programmation dynamique) mais le pire cas reste exponentiel.

\paragraph{Problème du voyageur de commerce (TSP, \(\mathcal{O}(n!)\)).}

Énoncé : trouver le plus court circuit passant une fois par chaque ville et revenant au départ.

Une approche brute force consiste à énumérer toutes les permutations possibles :

\begin{lstlisting}[language=Java]
public class TravelingSalesman {

    static class Result {
        java.util.List<Integer> tour;
        int distance;

        Result(java.util.List<Integer> tour, int distance) {
            this.tour = tour;
            this.distance = distance;
        }
    }

    public static Result solve(int[][] distanceMatrix) {
        boolean[] visited = new boolean[distanceMatrix.length];
        visited[0] = true; // ville de départ fixée à 0

        java.util.List<Integer> current = new java.util.ArrayList<>();
        current.add(0);

        return bestTour(visited, current, 0, distanceMatrix);
    }

    private static Result bestTour(boolean[] visited,
                                   java.util.List<Integer> current,
                                   int currentLength,
                                   int[][] d) {
        int last = current.get(current.size() - 1);

        if (current.size() == visited.length) {
            int total = currentLength + d[last][0];
            return new Result(new java.util.ArrayList<>(current), total);
        }

        Result best = null;
        for (int city = 0; city < visited.length; city++) {
            if (!visited[city]) {
                visited[city] = true;
                current.add(city);

                Result cand = bestTour(visited, current,
                                       currentLength + d[last][city], d);

                if (best == null || cand.distance < best.distance) {
                    best = cand;
                }

                current.remove(current.size() - 1);
                visited[city] = false;
            }
        }
        return best;
    }
}
\end{lstlisting}

La complexité est factorielle : \(\mathcal{O}(n!)\). Dès \(n \gtrsim 12\), cette approche devient impraticable.

\paragraph{Exercice (bitCount).}

Déterminer la complexité temporelle (meilleur/pire cas) de :

\begin{lstlisting}[language=Java]
/**
 * Compte le nombre de bits nécessaires pour représenter n (> 0).
 */
public static int bitCount(int n) {
    int bitCount = 0;
    while (n > 0) {
        bitCount++;
        n = n >> 1; // division par 2
    }
    return bitCount;
}
\end{lstlisting}

%------------------------------------------------------------
\section{Complexité spatiale}
\label{sec:space-complexity}

La \emph{complexité spatiale} mesure la quantité de mémoire utilisée par un algorithme en fonction de la taille de
l’entrée.

On distingue :

\begin{itemize}
  \item l’\textbf{espace d’entrée} : mémoire occupée par les paramètres et les données déjà existantes (par exemple, le
        tableau passé à une méthode) ;
  \item l’\textbf{espace auxiliaire} : mémoire supplémentaire allouée par l’algorithme pendant son exécution
        (variables locales, tableaux temporaires, pile d’appels, \dots).
\end{itemize}

Par définition :
\[
\text{complexité spatiale} =
\text{espace auxiliaire} + \text{espace d'entrée}.
\]

\paragraph{Exemple \texttt{sum}.}

\begin{lstlisting}[language=Java]
public static int sum(int[] array) {
    int sum = 0;              // O(1)
    for (int i = 0; i < array.length; i++) {
        sum += array[i];
    }
    return sum;
}
\end{lstlisting}

\begin{itemize}
  \item espace auxiliaire : \(\mathcal{O}(1)\),
  \item espace d’entrée : \(\Theta(n)\) pour le tableau,
\end{itemize}

d’où une complexité spatiale \(\Theta(n)\).

\paragraph{Exemple \texttt{range}.}

\begin{lstlisting}[language=Java]
public static int[] range(int n) {
    int[] result = new int[n];
    for (int i = 0; i < n; i++) {
        result[i] = i;
    }
    return result;
}
\end{lstlisting}

\begin{itemize}
  \item espace auxiliaire : \(\Theta(n)\) (le tableau créé),
  \item espace d’entrée : \(\mathcal{O}(1)\) (un entier).
\end{itemize}

On peut exprimer la complexité en fonction de \(|n|\) ou de la taille en bits de la représentation ; il est important de
préciser ce que représente \(n\).

\subsection{Complexité spatiale des algorithmes récursifs}

Pour un algorithme récursif, l’espace auxiliaire inclut aussi la \emph{pile d’appels} : à chaque appel récursif, le contexte
(local) est empilé.

Exemple : factorielle.

\begin{lstlisting}[language=Java]
public class Factorial {

    // Version récursive
    public static long factorialRec(int n) {
        if (n == 0) return 1L;
        return n * factorialRec(n - 1);
    }

    // Version itérative
    public static long factorialIter(int n) {
        long result = 1L;
        for (int i = 1; i <= n; i++) {
            result *= i;
        }
        return result;
    }
}
\end{lstlisting}

Les deux versions ont une complexité temporelle \(\Theta(n)\).

\begin{itemize}
  \item Version itérative : espace auxiliaire \(\mathcal{O}(1)\).
  \item Version récursive : espace auxiliaire \(\Theta(n)\) (pile d’appels de profondeur \(n\)).
\end{itemize}

Ainsi, même sans tableau explicite, la récursivité consomme de la mémoire.

\paragraph{Améliorer l’espace de Merge Sort.}

Dans l’implémentation naïve de \texttt{mergeSort}, on crée de nouveaux tableaux à chaque niveau récursif, ce qui mène
à un espace auxiliaire total \(\mathcal{O}(n \log n)\). On peut l’améliorer en réutilisant un unique tableau temporaire :

\begin{lstlisting}[language=Java]
public class MergeSortOptimized {

    public void sort(int[] arr) {
        int[] temp = new int[arr.length];
        sort(arr, temp, 0, arr.length - 1);
    }

    private void sort(int[] arr, int[] temp, int left, int right) {
        if (left >= right) return;
        int mid = (left + right) / 2;
        sort(arr, temp, left, mid);
        sort(arr, temp, mid + 1, right);
        merge(arr, temp, left, mid, right);
    }

    private void merge(int[] arr, int[] temp,
                       int left, int mid, int right) {
        int i = left;
        int j = mid + 1;
        int k = left;

        while (i <= mid && j <= right) {
            if (arr[i] <= arr[j]) temp[k++] = arr[i++];
            else                  temp[k++] = arr[j++];
        }
        while (i <= mid)   temp[k++] = arr[i++];
        while (j <= right) temp[k++] = arr[j++];

        for (int t = left; t <= right; t++) {
            arr[t] = temp[t];
        }
    }
}
\end{lstlisting}

On obtient ainsi une complexité spatiale auxiliaire \(\mathcal{O}(n)\) au lieu de \(\mathcal{O}(n \log n)\).

\bigskip

\noindent
En général, la complexité temporelle est au moins de l’ordre de l’espace auxiliaire : manipuler \(k\) cellules mémoire
demande au moins \(k\) opérations.

%------------------------------------------------------------
\section{Correction des algorithmes}
\label{sec:algo-correctness}

\subsection{Invariants de boucle}

Un \emph{invariant de boucle} est une propriété vraie :

\begin{enumerate}
  \item avant la première itération (initialisation),
  \item avant chaque itération (maintenance),
  \item à la fin de la boucle (terminaison).
\end{enumerate}

Exemple : maximum d’un tableau.

\begin{lstlisting}[language=Java]
/**
 * Maximum avec une boucle while.
 */
public static int maxWhile(int[] a) {
    int m = a[0];
    int i = 1;
    // inv : m est le max de a[0..i-1]
    while (i < a.length) {
        if (a[i] > m) {
            m = a[i];
        }
        i++;
        // inv maintenu : m est le max de a[0..i-1]
    }
    // m est le max de a[0..a.length-1]
    return m;
}
\end{lstlisting}

Même idée avec une boucle \texttt{for} :

\begin{lstlisting}[language=Java]
public static int max(int[] a) {
    int m = a[0];
    // inv : m est le max de a[0..0]
    for (int i = 1; i < a.length; i++) {
        // inv : m est le max de a[0..i-1]
        if (a[i] > m) {
            m = a[i];
        }
        // inv : m est le max de a[0..i]
    }
    return m; // max global
}
\end{lstlisting}

On prouve la correction en montrant :

\begin{itemize}
  \item \textbf{Initialisation} : l’invariant est vrai avant la première itération.
  \item \textbf{Maintenance} : s’il est vrai au début d’une itération, il reste vrai après.
  \item \textbf{Terminaison} : combiné à la condition de sortie, il implique la post-condition voulue.
\end{itemize}

\subsection{Preuve de correction récursive}

Pour un algorithme récursif, on raisonne par induction.

Exemple : maximum récursif d’un préfixe.

\begin{lstlisting}[language=Java]
/**
 * Max de a[0..i].
 */
private static int maxRecur(int[] a, int i) {
    if (i == 0) {
        return a[0];
    } else {
        return Math.max(maxRecur(a, i - 1), a[i]);
    }
}
\end{lstlisting}

\begin{itemize}
  \item \textbf{Base} : pour \(i = 0\), on renvoie \(a[0]\), correct.
  \item \textbf{Induction} : on suppose correct pour \(i-1\). Alors
        \(\max(a[0..i]) = \max(\max(a[0..i-1]), a[i])\), ce que calcule la fonction.
\end{itemize}

\paragraph{Exercices.}

\begin{itemize}
  \item Proposer un invariant pour l’algorithme de sous-tableau de somme maximale (\texttt{maxSubArray}) et le prouver.
  \item Proposer un invariant pour la boucle externe de \texttt{bubbleSort} :
\end{itemize}

\begin{lstlisting}[language=Java]
public static void bubbleSort(int[] array) {
    int n = array.length;
    for (int i = 0; i < n - 1; i++) {
        // invariant ?
        for (int j = 0; j < n - i - 1; j++) {
            if (array[j] > array[j + 1]) {
                int tmp = array[j];
                array[j] = array[j + 1];
                array[j + 1] = tmp;
            }
        }
    }
}
\end{lstlisting}

%------------------------------------------------------------
\section{Types de données abstraits (ADT)}
\label{sec:adt}

Un \emph{type de données abstrait} (ADT) décrit un ensemble de valeurs et d’opérations \emph{sans} imposer la
représentation interne.

Exemple : \texttt{List<E>}.

\begin{itemize}
  \item Opérations typiques : \texttt{add}, \texttt{get}, \texttt{remove}, \texttt{size}, \dots
  \item Implémentations possibles : \texttt{ArrayList}, \texttt{LinkedList}, \dots
\end{itemize}

Le contrat logique reste le même, même si les performances diffèrent.

\subsection*{Génériques}

\begin{lstlisting}[language=Java]
List<String> fruits = new java.util.LinkedList<>();
fruits.add("Apple");
fruits.add("Banana");
fruits.add("Cherry");
fruits.remove("Banana");
\end{lstlisting}

Les génériques (\texttt{<T>}) permettent de paramétrer les types et évitent les cast dangereux.

%------------------------------------------------------------
\subsection{Stack ADT}
\label{sec:stack-adt}

Une pile (stack) suit le principe LIFO.

\begin{lstlisting}[language=Java]
public interface Stack<T> {
    void push(T item);  // empile
    T pop();            // dépile et renvoie
    T peek();           // lit le sommet
    boolean isEmpty();
    int size();
}
\end{lstlisting}

\subsubsection*{Implémentation chaînée}

\begin{lstlisting}[language=Java]
public class LinkedStack<T> implements Stack<T> {

    private static class Node<T> {
        T item;
        Node<T> next;
        Node(T item, Node<T> next) {
            this.item = item;
            this.next = next;
        }
    }

    private Node<T> top;
    private int size;

    @Override
    public void push(T item) {
        top = new Node<>(item, top);
        size++;
    }

    @Override
    public T pop() {
        if (isEmpty()) throw new RuntimeException("Stack is empty");
        T item = top.item;
        top = top.next;
        size--;
        return item;
    }

    @Override
    public T peek() {
        if (isEmpty()) throw new RuntimeException("Stack is empty");
        return top.item;
    }

    @Override
    public boolean isEmpty() {
        return top == null;
    }

    @Override
    public int size() {
        return size;
    }
}
\end{lstlisting}

Toutes les opérations de pile sont en temps \(\mathcal{O}(1)\).

\subsubsection*{Implémentation par tableau dynamique}

\begin{lstlisting}[language=Java]
public class DynamicArrayStack<T> implements Stack<T> {

    private T[] array;
    private int top; // index du sommet
    @SuppressWarnings("unchecked")
    public DynamicArrayStack(int initialCapacity) {
        array = (T[]) new Object[initialCapacity];
        top = -1;
    }

    @Override
    public void push(T item) {
        if (top == array.length - 1) {
            resize(2 * array.length);
        }
        array[++top] = item;
    }

    @Override
    public T pop() {
        if (isEmpty()) throw new RuntimeException("Stack is empty");
        T item = array[top];
        array[top--] = null;
        if (top > 0 && top == array.length / 4) {
            resize(array.length / 2);
        }
        return item;
    }

    @Override
    public T peek() {
        if (isEmpty()) throw new RuntimeException("Stack is empty");
        return array[top];
    }

    @Override
    public boolean isEmpty() {
        return top == -1;
    }

    @Override
    public int size() {
        return top + 1;
    }

    private void resize(int newCapacity) {
        @SuppressWarnings("unchecked")
        T[] newArray = (T[]) new Object[newCapacity];
        System.arraycopy(array, 0, newArray, 0, top + 1);
        array = newArray;
    }
}
\end{lstlisting}

Les opérations sont \(\mathcal{O}(1)\) en temps amorti grâce au redimensionnement exponentiel.

\subsubsection*{Évaluation d’expressions avec deux piles}

On peut évaluer une expression entièrement parenthésée à l’aide :

\begin{itemize}
  \item d’une pile d’opérateurs (\texttt{ops}),
  \item d’une pile de valeurs (\texttt{vals}).
\end{itemize}

Idée : à chaque \texttt{")"}, on dépile un opérateur et les deux dernières valeurs, on applique, et on réempile le
résultat. La complexité est \(\mathcal{O}(n)\).

L’invariant informel :

\begin{itemize}
  \item \texttt{vals} contient les valeurs (ou sous-résultats) des sous-expressions déjà complètement évaluées ;
  \item \texttt{ops} contient les opérateurs en attente de leur opérande droit.
\end{itemize}

\paragraph{Exercice.} Écrire une version récursive qui évalue une expression parenthésée en s’appuyant sur la pile
d’appels plutôt que sur des piles explicites.

%------------------------------------------------------------
\subsection{Arbres}
\label{sec:trees}

Les arbres permettent de représenter des données hiérarchiques : systèmes de fichiers, arbres généalogiques,
expressions arithmétiques, etc.

\subsubsection*{Arbre binaire simple}

\begin{lstlisting}[language=Java]
public class LinkedBinaryTree {

    private Node root;

    class Node {
        int val;
        Node left;
        Node right;
        Node(int val) { this.val = val; }
    }

    public static LinkedBinaryTree leaf(int val) {
        LinkedBinaryTree t = new LinkedBinaryTree();
        t.root = t.new Node(val);
        return t;
    }

    public static LinkedBinaryTree combine(int val,
                                           LinkedBinaryTree left,
                                           LinkedBinaryTree right) {
        LinkedBinaryTree t = new LinkedBinaryTree();
        t.root = t.new Node(val);
        t.root.left = left.root;
        t.root.right = right.root;
        return t;
    }
}
\end{lstlisting}

\begin{figure}[h]
  \centering
  \includegraphics[width=.5\linewidth]{chap4_imgs/chap4_p112_img1.png}
  \caption{Exemple d’arbre binaire.}
\end{figure}

\paragraph{Parcours.}

\begin{itemize}
  \item Pré-ordre : noeud, gauche, droite.
  \item En-ordre : gauche, noeud, droite.
  \item Post-ordre : gauche, droite, noeud.
\end{itemize}

Chaque parcours visite chaque noeud une fois : \(\Theta(n)\).

\subsubsection*{Arbre d’expressions binaires}

On peut modéliser une expression comme un arbre binaire :

\begin{lstlisting}[language=Java]
public abstract class BinaryExpressionTree {

    abstract int evaluate();
    abstract String prettyPrint();

    public BinaryExpressionTree mul(BinaryExpressionTree right) {
        return new OperatorExpressionTree(this, right, '*');
    }
    public BinaryExpressionTree plus(BinaryExpressionTree right) {
        return new OperatorExpressionTree(this, right, '+');
    }
    // etc.

    private static class OperatorExpressionTree extends BinaryExpressionTree {
        private final BinaryExpressionTree left;
        private final BinaryExpressionTree right;
        private final char op;

        OperatorExpressionTree(BinaryExpressionTree left,
                               BinaryExpressionTree right,
                               char op) {
            this.left = left;
            this.right = right;
            this.op = op;
        }

        @Override
        int evaluate() {
            int lv = left.evaluate();
            int rv = right.evaluate();
            return switch (op) {
                case '+' -> lv + rv;
                case '-' -> lv - rv;
                case '*' -> lv * rv;
                case '/' -> lv / rv;
                default -> throw new IllegalArgumentException();
            };
        }

        @Override
        String prettyPrint() {
            return "(" + left.prettyPrint() + op + right.prettyPrint() + ")";
        }
    }

    private static class ValueExpressionTree extends BinaryExpressionTree {
        private final int value;
        ValueExpressionTree(int value) { this.value = value; }

        @Override
        int evaluate() { return value; }

        @Override
        String prettyPrint() { return Integer.toString(value); }
    }

    public static BinaryExpressionTree value(int v) {
        return new ValueExpressionTree(v);
    }
}
\end{lstlisting}

\paragraph{Exercice.} Ajouter \texttt{rpnPrint()} qui affiche l’expression en notation polonaise inversée.

\subsubsection*{Ensembles avec arbre binaire de recherche (BST)}

On définit un ADT pour un ensemble d’entiers sans doublon :

\begin{lstlisting}[language=Java]
public interface IntSet {
    void add(int val);
    boolean contains(int val);
}
\end{lstlisting}

Implémentation avec un \emph{binary search tree} (BST) :

\begin{lstlisting}[language=Java]
public class BinarySearchTree implements IntSet {

    private class Node {
        int val;
        Node left, right;
        Node(int val) { this.val = val; }
    }

    private Node root;

    @Override
    public void add(int val) {
        root = addRec(root, val);
    }

    private Node addRec(Node current, int val) {
        if (current == null) return new Node(val);
        if (val < current.val) current.left = addRec(current.left, val);
        else if (val > current.val) current.right = addRec(current.right, val);
        // sinon déjà présent
        return current;
    }

    @Override
    public boolean contains(int val) {
        return containsRec(root, val);
    }

    private boolean containsRec(Node current, int val) {
        if (current == null) return false;
        if (val == current.val) return true;
        if (val < current.val) return containsRec(current.left, val);
        else                    return containsRec(current.right, val);
    }
}
\end{lstlisting}

Si l’arbre reste équilibré, les opérations sont en \(\mathcal{O}(\log n)\).

%------------------------------------------------------------
\subsection{Maps}
\label{sec:maps}

Un \texttt{Map<K,V>} généralise le tableau : au lieu d’indices entiers, on a des \emph{clés} arbitraires.

Exemple avec \texttt{Hashtable} :

\begin{lstlisting}[language=Java]
import java.util.Hashtable;
import java.util.Map;

public class MapExample {
    public static void main(String[] args) {
        Map<String,Integer> treeHeights = new Hashtable<>();
        treeHeights.put("Oak", 20);
        treeHeights.put("Pine", 25);
        System.out.println(treeHeights.get("Oak"));
    }
}
\end{lstlisting}

Principe de la table de hachage :

\begin{itemize}
  \item une fonction de hachage associe à une clé un entier (hash code),
  \item on ramène ce hash dans \([0, N-1]\) pour indexer un tableau,
  \item en cas de collisions, on stocke une liste chaînée d’entrées à cette case.
\end{itemize}

Insertion et recherche ont une complexité moyenne \(\mathcal{O}(1)\) si le \emph{load factor} est maîtrisé.

\begin{lstlisting}[language=Java]
public class SimpleHashtable {

    private static final int N = 10;

    private static class Entry {
        final String key;
        Integer value;
        Entry(String key, Integer value) {
            this.key = key;
            this.value = value;
        }
    }

    @SuppressWarnings("unchecked")
    private java.util.LinkedList<Entry>[] table =
        new java.util.LinkedList[N];

    public SimpleHashtable() {
        for (int i = 0; i < N; i++) {
            table[i] = new java.util.LinkedList<>();
        }
    }

    private int hash(String key) {
        return Math.abs(key.hashCode()) % N;
    }

    public void put(String key, Integer value) {
        int index = hash(key);
        var bucket = table[index];
        for (Entry e : bucket) {
            if (e.key.equals(key)) {
                e.value = value;
                return;
            }
        }
        bucket.add(new Entry(key, value));
    }

    public Integer get(String key) {
        int index = hash(key);
        var bucket = table[index];
        for (Entry e : bucket) {
            if (e.key.equals(key)) return e.value;
        }
        return null;
    }
}
\end{lstlisting}

%------------------------------------------------------------
\section{Itérateurs}
\label{sec:iterators}

Un \emph{itérateur} est un objet qui parcourt une collection sans exposer sa représentation interne.

Interface standard :

\begin{lstlisting}[language=Java]
public interface Iterator<E> {
    boolean hasNext();
    E next();
    // remove() optionnel
}
\end{lstlisting}

Exemple d’utilisation :

\begin{lstlisting}[language=Java]
ArrayList<String> list = new ArrayList<>();
list.add("A"); list.add("B"); list.add("C");

Iterator<String> it = list.iterator();
while (it.hasNext()) {
    String e = it.next();
    System.out.println(e);
}
\end{lstlisting}

L’interface \texttt{Iterable<T>} fournit la méthode \texttt{iterator()} et permet la boucle \texttt{for-each} :

\begin{lstlisting}[language=Java]
for (String e : list) {
    System.out.println(e);
}
\end{lstlisting}

\subsection{Implémenter ses propres itérateurs}
\label{sec:custom-iterators}

Pour rendre un ADT itérable, on le fait étendre \texttt{Iterable<T>} :

\begin{lstlisting}[language=Java]
public interface Stack<T> extends Iterable<T> {
    void push(T item);
    T pop();
    T peek();
    boolean isEmpty();
    int size();
}
\end{lstlisting}

\subsubsection*{Itérateur fail-fast pour \texttt{LinkedStack}}

On suit la stratégie classique des collections Java : un compteur de modifications (\texttt{modCount}) est comparé à
une copie figée dans l’itérateur.

\begin{lstlisting}[language=Java]
import java.util.Iterator;
import java.util.ConcurrentModificationException;

public class LinkedStack<T> implements Stack<T> {

    private static class Node<T> {
        T item;
        Node<T> next;
        Node(T item, Node<T> next) {
            this.item = item;
            this.next = next;
        }
    }

    private Node<T> top;
    private int size = 0;
    private int modCount = 0;

    @Override
    public void push(T item) {
        top = new Node<>(item, top);
        size++;
        modCount++;
    }

    @Override
    public T pop() {
        if (isEmpty()) throw new RuntimeException("Stack is empty");
        T item = top.item;
        top = top.next;
        size--;
        modCount++;
        return item;
    }

    @Override
    public T peek() {
        if (isEmpty()) throw new RuntimeException("Stack is empty");
        return top.item;
    }

    @Override
    public boolean isEmpty() { return top == null; }

    @Override
    public int size() { return size; }

    @Override
    public Iterator<T> iterator() {
        return new LinkedStackIterator();
    }

    private class LinkedStackIterator implements Iterator<T> {

        private Node<T> current = top;
        private final int expectedModCount = modCount;

        @Override
        public boolean hasNext() {
            if (expectedModCount != modCount) {
                throw new ConcurrentModificationException();
            }
            return current != null;
        }

        @Override
        public T next() {
            if (expectedModCount != modCount) {
                throw new ConcurrentModificationException();
            }
            if (current == null) {
                throw new IllegalStateException("No more items");
            }
            T item = current.item;
            current = current.next;
            return item;
        }
    }
}
\end{lstlisting}

Toute modification structurale de la pile après création de l’itérateur provoque une
\texttt{ConcurrentModificationException} au prochain \texttt{hasNext()} / \texttt{next()}.

% Fin du chapitre 4
\chapter{Programmation parallèle}
\label{chap:parallel}




\end{document}
